%%% Defined here (in order): mathematical macros, theorem style, colour of hyperlinks, set spaces,
%%% tikz, reminder, fix for bigint package

%% Mathematical macros %%

% Logic
\newcommand{\eq}{\Leftrightarrow}
\newcommand{\imp}{\Rightarrow}
\newcommand{\To}{\longrightarrow}
\newcommand{\inj}{\hookrightarrow}
\newcommand{\surj}{\longrightarrow\!\!\!\!\to}
\newcommand{\isom}{\cong}

% Sets
\newcommand{\N}{\mathbb{N}}
\newcommand{\Z}{\mathbb{Z}}
\newcommand{\Q}{\mathbb{Q}}
\newcommand{\R}{\mathbb{R}}
\newcommand{\C}{\mathbb{C}}
\newcommand{\K}{\mathbb{K}}
\renewcommand{\P}{\mathbb{P}}

% Operators
\DeclarePairedDelimiter\ceil{\lceil}{\rceil}
\DeclarePairedDelimiter\floor{\lfloor}{\rfloor}
\newcommand\set[1]{\left\{#1\right\}}
\newcommand\abs[1]{\left|#1\right|}
\DeclareMathOperator{\id}{id}
\DeclareMathOperator{\atanh}{atanh}
\DeclareMathOperator{\asinh}{asinh}
\DeclareMathOperator{\rk}{rk}
\DeclareMathOperator{\sgn}{sgn}
\DeclareMathOperator{\Jac}{Jac}
\DeclareMathOperator{\Aut}{Aut}
\DeclareMathOperator{\GL}{GL}
\DeclareMathOperator{\PSl}{PSL}
\DeclareMathOperator{\pr}{pr} % covering map
\DeclareMathOperator{\ord}{ord}
\renewcommand{\Re}{\operatorname{Re}}
\renewcommand{\Im}{\operatorname{Im}}
\renewcommand{\div}{\operatorname{div}}
\newcommand{\Div}{\operatorname{Div}}
\renewcommand{\deg}{\operatorname{deg}}
\newcommand{\Prin}{\operatorname{Prin}}
\newcommand{\supp}{\operatorname{supp}}
\renewcommand{\ord}{\operatorname{ord}}

% Variables
\newcommand{\cu}{\mathcal{C}} % projective curve
\newcommand{\caff}{\mathcal{C}_{\text{aff}}} % affine model
\newcommand{\cafft}{\tilde{\mathcal{C}}_{\text{aff}}} % affine model 2
\newcommand{\X}{\{  x_1,\dots,x_d \}} % branch points
\renewcommand\d{\mathrm{d}\,}
\newcommand{\dx}{\mathrm d x}
\newcommand{\dy}{\mathrm d y}
\newcommand{\dt}{\mathrm d t}
\newcommand{\du}{\mathrm d u}
\newcommand{\w}{\omega} % differential
\newcommand{\wtij}{\omega_{\tilde{i},j}}
\newcommand{\wij}{\frac{\dx^i}{iy^j}} %w \in W
\newcommand{\W}{\mathcal{W}} % basis of holomorphic differentials
\newcommand{\WM}{\mathcal{W}^{\text{mer}}} % basis of meromorphic differentials
\newcommand{\hd}{\Omega^1_{\mathcal{C}}} % holomorphic differentials
\newcommand{\coho}{H^1_{\text{dR}}(\cu)} % holomorphic differentials in deRham
\newcommand{\homo}{H_1(\cu,\Z)} % first integral homology group
\newcommand{\cyab}{\gamma_{a,b}}
\newcommand{\cycd}{\gamma_{c,d}}
\newcommand{\cyad}{\gamma_{a,d}}
\newcommand{\cybd}{\gamma_{b,d}}
\newcommand{\cyabl}{\gamma_{a,b}^{(l)}}
\newcommand{\cyabk}{\gamma_{a,b}^{(k)}}
\newcommand{\cycdk}{\gamma_{c,d}^{(k)}}
\newcommand{\cycdl}{\gamma_{c,d}^{(l)}}
\newcommand{\cyt}{\tilde{gamma}}
\newcommand{\cytab}{\tilde{gamma}(a,b)}
\newcommand{\cytcd}{\tilde{gamma}(c,d)}
\newcommand{\OA}{\Omega_A} % A-periods
\newcommand{\OB}{\Omega_B} % B-periods
\newcommand{\OC}{\Omega_{\Gamma}} % C-periods
\newcommand{\AJ}{\mathcal{A}} % Abel-Jacobi map
\newcommand{\sab}{s(\alpha,\beta)} % shifting number
\newcommand{\sij}{s(\alpha^{(i)},\beta^{(j)})} % shifting number
\newcommand{\yab}{y_{a,b}}
\newcommand{\ycd}{y_{c,d}}
\newcommand{\yaxp}{y_{a,x_P}}
\newcommand{\mr}{^{\frac{1}{m}}}
\newcommand{\vab}{V_{a,b}}
\newcommand{\vbd}{V_{b,d}}
\newcommand{\vcd}{V_{c,d}}
\newcommand{\uab}{u_{a,b}}
\newcommand{\ubd}{u_{b,d}}
\newcommand{\ucd}{u_{c,d}}
\newcommand{\uaxp}{u_{a,x_P}}
\newcommand{\yt}{\tilde{y}}
\newcommand{\ytab}{\tilde{y}_{a,b}}
\newcommand{\ytcd}{\tilde{y}_{c,d}}
\newcommand{\ytaxp}{\tilde{y}_{a,x_P}}
\newcommand{\xt}{\tilde{x}}
\DeclareMathOperator{\dist}{dist}

% complexity
\newcommand\cmul{\mathcal M}
\newcommand\ctrig{\mathcal T}
\newcommand\ctot{\mathcal E}


% Pascal
\renewcommand\d{\mathrm{d}}
\newcommand\phiab{\phi_{a,b}}
\DeclareUnicodeCharacter{3B1}{\alpha}
\DeclareUnicodeCharacter{3BB}{\lambda}
\DeclareUnicodeCharacter{3C4}{\tau}
\DeclareUnicodeCharacter{3C0}{\pi}
\DeclareUnicodeCharacter{3B7}{\eta}
\DeclareUnicodeCharacter{3C6}{\varphi}
\DeclareUnicodeCharacter{3B1}{\alpha}
\DeclareUnicodeCharacter{3B2}{\beta}
\DeclareUnicodeCharacter{3BB}{\lambda}
\DeclareUnicodeCharacter{3B3}{\gamma}
\DeclareUnicodeCharacter{3B4}{\delta}
\DeclareUnicodeCharacter{3C4}{\tau}
\DeclareUnicodeCharacter{3C0}{\pi}
\DeclareUnicodeCharacter{3B5}{\varepsilon}
\DeclareUnicodeCharacter{3C9}{\omega}
\DeclareUnicodeCharacter{3C1}{\rho}


%%  Theorem style %%
\theoremstyle{plain}
\newtheorem{thm}{Theorem}[section]
\newtheorem{prop}[thm]{Proposition}
\newtheorem{lemma}[thm]{Lemma}
\newtheorem{coro}[thm]{Corollary}

\theoremstyle{definition}
\newtheorem{defn}[thm]{Definition}
\newtheorem{ex}[thm]{Example}
\newtheorem{alg}[thm]{Algorithm}

\theoremstyle{remark}
\newtheorem{rmk}[thm]{Remark}


%% Colour of hyperlinks %%
\hypersetup{linkbordercolor  ={1 1 1}}
\hypersetup{citebordercolor  ={1 1 1}}
\hypersetup{urlbordercolor  ={1 1 1}}


%% Set spaces%%
\setlength\abovedisplayshortskip{5pt}
\setlength\belowdisplayshortskip{5pt}
\setlength\abovedisplayskip{5pt}
\setlength\belowdisplayskip{5pt}
% \setlength{\parskip}{0.75\baselineskip}
% \setlength{\parindent}{0em}
\newcommand{\abstand}{{\\[0.25\baselineskip]}}
\newcommand{\abstandl}{{\\[0.5\baselineskip]}}
\newcommand{\abstandll}{{\\[0.75\baselineskip]}}


%% Tikz %%
\usetikzlibrary{arrows}
\usetikzlibrary{shapes.misc}
\usetikzlibrary{decorations.markings}
\tikzset{cross/.style={cross out, draw=black, minimum size=2*(#1-\pgflinewidth), inner sep=0pt, outer sep=0pt},cross/.default={1pt}}
\tikzset{
    halfarrow1/.style={postaction={decorate},
        decoration={markings,mark=at position .5 with
        {\arrow[line width=0.4mm]{>}}}} }
\tikzset{
    halfarrow2/.style={postaction={decorate},
        decoration={markings,mark=at position .5 with
        {\arrow[line width=0.4mm]{<}}}} }


%% Reminder %%
\newcommand{\todo}{{\color{red}\textbf{TODO}}:\;}


%% Problem with lmodern and bitint packages %%
\DeclareFontFamily{OMX}{lmex}{}
\DeclareFontShape{OMX}{lmex}{m}{n}{<-> lmex10}{}
