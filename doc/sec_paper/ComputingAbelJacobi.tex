\documentclass[main.tex]{subfiles}

\begin{document}

  \section{Computing the Abel-Jacobi map}\label{sec:comp_ajm}
  
   %Assume for this section that we already computed a period matrix following the strategy described in Section \ref{m-sec:strat_pm}.
   Here we are concerned with explicitly computing the Abel-Jacobi map of degree zero divisors, for a general introduction see Section \ref{m-sec:ajm}.
   
   Let $D = \sum_{P \in \cu} a_P P \in \Div^0(\cu)$. Choosing a basepoint $P_0 \in \cu$, the computation of $\AJ$ reduces to (using linearity)
   \begin{equation}
     \AJ([D]) \equiv \sum_{P \in \cu} a_P \int_{P_0}^P \bar\w \mod \Lambda.
   \end{equation}
  For every $P \in \cu$, $\int_{P_0}^P \bar\w$ is a linear combination of vector integrals of the form
  \begin{align*}
    \int_{P_0}^{P_k} \bar\w\quad \text{(see \S \ref{m-subsec:ajm_bp}),} \quad
    \int_{P_k}^{P} \bar\w \quad \text{(see \S \ref{m-subsec:ajm_finite})}
    \quad \text{and} \quad \int_{P_0}^{P_{\infty}} \bar\w\quad \text{(see \S \ref{m-subsec:ajm_infty}),} \quad \text{where}
  \end{align*}
  \begin{itemize}
   \item $P = (x_P,y_P) \in \cu$ is a finite point on the curve,
   \item $P_k = (x_k,0) \in \cu$ is a ramification point, i.e.\ $x_k \in X$, and
   \item $P_{\infty} \in \cu$ is a point above infinity.
  \end{itemize}
  Finally, the resulting vector integral has to reduced modulo the period lattice, which is covered in \S \ref{m-subsec:lat_red}.
  
  Assume for this section that we already computed a period matrix (and all related data) following the Strategy from Section \ref{m-sec:strat_pm}. 
  
  Typically, we choose as basepoint the ramification point $P_0 = (x_0,0)$, where $x_0 \in X$ is the root of the spanning tree $T = (X,E)$.

  
   
  \subsection{Between ramification points}\label{subsec:ajm_bp}

  Consider a path $(x_0=x_{k_0},x_{k_1},\dots,x_{k_{n-1}},x_{k_t}=x_k)$ in the spanning tree connecting $x_0$ and $x_k$. Then we have that
  \begin{align}
    \int_{P_0}^{P_k} \bar\w = \sum_{j = 0}^{t-1}  \int_{P_{k_j}}^{P_{k_{j+1}}} \bar\w.
  \end{align}
  Denote $a = x_{k_j}, b = x_{k_{j+1}} \in X$. From \S \ref{m-subsec:cycles_homo} we know that for $(a,b) \in E$ a smooth path between $P_a=(a,0)$ and $P_b=(b,0)$ is given by
  \begin{align*}
   \gamma = \{  (x,\yab(x))  \mid  x \in [a,b]  \}.
  \end{align*}
  Fix a differential $\w_{i,j} \in \W$. Then, according to the proof of Theorem \ref{m-thm:periods} the corresponding integral is just an elementary integral
  \begin{align}
   \int_{\gamma} \w_{i,j}  & =  C_{i,j}  \int_{-1}^1 \frac{\varphi_{i,j}(u)}{(1-u^2)^{\frac{j}{m}}} \du,
  \end{align}
  which has already been evaluated during the period matrix computation.


  \subsection{Reaching non-ramification points}\label{subsec:ajm_finite}

   \todo Rework
   Let $P = (x_P,y_P) \in \cu$ and $a \in X$ such that $X\cap]a,x_P]=\varnothing$. As in \S \ref{m-subsec:roots_branches} we define a locally analytic branch of $\cu$ via the functions
  \begin{align}\label{def:yax}
  \begin{split}
   \yaxp(x)  & =  \left(\frac{x_P-a}{2}\right)^{\frac{d}{m}}  (1 + u(x))^{\frac{j}{m}}  \ytaxp(u(x)), \quad \text{with} \\
   \ytaxp(u(x))  & =  \prod_{x_k \ne a} (u(x) - u(x_k))\mr \quad \text{and} \\
   u(x) & =  u_{a,x_P}(x) = \frac{2x-x_P-a}{x_P-a}.
   \end{split}
  \end{align}
  Now $\yaxp(x)$ is a branch of $\cu$ that is holomorphic in a neighbourhood of $]a,x_P]$.

  Define a path
  \begin{align*}
   \gamma_{a,x_P} = \{  (x,\yaxp(x))  \mid  x \in [a,x_P]  \}.
  \end{align*}
  Now, we can compute a \emph{shifting number} at the endpoint $x=x_P$
  \begin{align*}
   \zeta^s & = \frac{y_P}{\yaxp(x_P)} = \frac{y_P}{\left(\frac{x_P-a}{2}\right)^{\frac{d}{m}} 2\mr \ytaxp(x_P)} \\
   \Rightarrow s & = \frac{m}{2\pi} \arg\left(  \frac{y_P}{ (x_P-a)^{\frac{d}{m}} \ytaxp(x_P)} \right)
  \end{align*}
  and define a smooth path between $P_a$ and $P$ on $\cu$ by
  \begin{align*}
   \gamma_{P_a,P} = \{  (x, \zeta^s \yaxp(x))  \mid  x \in [a,x_P]  \}.
  \end{align*}



  \begin{thm}\label{thm:ajm_finite_int}
  Let $\w_{i,j} \in \WM$. With notation as above we have
 \begin{align}\label{eq:aj_int}
  \begin{split}
       \int_{P_a}^P \w_{i,j} & = \left(\frac{x_P-a}{2}\right)^{i-\frac{dj}{m}} \zeta^{-js} \int_{-1}^1
      \frac{\left(u + \frac{x_P+a}{x_P-a}\right)^{i-1}}{(1+u)^{\frac{j}{m}} \ytaxp(u)^j}  \du.
%      \ytaxp(u) & = \prod_{x_k \ne a} ( u - u_{a,x_P}(x_k) )\mr, \\
%      s & = \frac{m}{2\pi}  \arg \left( y_P \cdot (x_P-a)^{-\frac{d}{m}} \right).
   \end{split}
   \end{align}
  \end{thm}
  \begin{proof}
    \begin{align}
    \begin{split}
     \int_{P_a}^P \w_{i,j}  & =  \int_{\gamma_{P,a_P}} \frac{x^{i-1}}{y^j}  \dx  =  \zeta^{-sj} \int_a^{x_P} \frac{x^{i-1}}{\yaxp(x)^j}  \dx \\  & = 
     \zeta^{-sj} \left(\frac{x_P-a}{2}\right)^{i-1-\frac{dj}{m}}
     \int_a^{x_P} \frac{\left(u(x)+\frac{x_P+a}{x_P-a}\right)^{i-1}}{(1+u(x))^{\frac{j}{m}}\ytaxp(u(x))^j}  \dx \\
     & =  \zeta^{-sj} \left(\frac{x_P-a}{2}\right)^{i-\frac{dj}{m}}
     \int_{-1}^{1} \frac{\left(u+\frac{x_P+a}{x_P-a}\right)^{i-1}}{(1+u)^{\frac{j}{m}}\ytaxp(u)^j}  \du \\
    \end{split}
  \end{align}
  \end{proof}

  \subsection{From basepoint to infinity}\label{subsec:ajm_infty}

    If there is only one point $P_{\infty}$ above infinity, i.e. $\delta = \gcd(m,d) = 1$, we can easily obtain from data that has already been computed.

    Consider the Bézout identity $\mu m + \nu d = 1$, with $\mu,\nu \in \Z$. We use the following following princial divisors to express $P_{\infty}$ in terms of the finite ramification points
    modulo the periods lattice.
    \begin{align}
     \begin{split}
      &\div(y^{\nu}) ) =  \nu \sum_{k = 1}^d P_k - \nu d P_{\infty},\\
      &\div((x-x_0)^{\mu})  =  \mu m P_0 - \mu m P_{\infty} ,\\
      D  =  & \div(y^{\nu}(x-x_0)^{\mu})  = \nu \sum_{k = 1}^d P_k + \mu m P_0 - P_{\infty}
     \end{split}
    \end{align}
    Integrating from $P_0$ to $P_{\infty}$ is equivalent to computing the Abel-Jacobi map of the divisor $D_{\infty} =  P_{\infty} - P_0 $. Therefore,
    \begin{align}
     \AJ( [D_{\infty}] )  & \equiv  \AJ([D_{\infty} + D])  =  \AJ( [ \nu \sum_{k = 1}^d P_k + (\mu m - 1) P_0 ]  ) \\  & \equiv  \AJ( [ \nu \sum_{k = 1}^d P_k ] )
      \equiv   \nu \sum_{k=1}^d \int_{P_0}^{P_k} \bar\w \mod \Lambda.
    \end{align}

    \todo Add case $\delta > 1$.



  \subsection{Lattice reduction}\label{subsec:lat_red}

    Necessary? Yes. \todo Explain reduction.

\biblio
\end{document}
