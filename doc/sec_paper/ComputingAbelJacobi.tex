\documentclass[main.tex]{subfiles}

\begin{document}

  \section{Computing the Abel-Jacobi map}\label{sec:comp_ajm}
  
   %Assume for this section that we already computed a period matrix following the strategy described in Section \ref{m-sec:strat_pm}.
   Here we are concerned with explicitly computing the Abel-Jacobi map of degree zero divisors, for a general introduction see Section \ref{m-sec:ajm}.
   
   Let $D = \sum_{P \in \cu} v_P P \in \Div^0(\cu)$. Choosing a basepoint $P_0 \in \cu$, the computation of $\AJ$ reduces to (using linearity)
   \begin{equation}
     \AJ([D]) \equiv \sum_{P \in \cu} v_P \int_{P_0}^P \bar\w \mod \Lambda.
   \end{equation}
  For every $P \in \cu$, $\int_{P_0}^P \bar\w$ is a linear combination of vector integrals of the form
  \begin{align*}
    \int_{P_0}^{P_k} \bar\w\quad \text{(see \S \ref{m-subsec:ajm_bp}),} \quad
    \int_{P_k}^{P} \bar\w \quad \text{(see \S \ref{m-subsec:ajm_finite})}
    \quad \text{and} \quad \int_{P_0}^{P_{\infty}} \bar\w\quad \text{(see \S \ref{m-subsec:ajm_infty}),} \quad \text{where}
  \end{align*}
  \begin{itemize}
   \item $P = (x_P,y_P) \in \cu$ is a finite point on the curve,
   \item $P_k = (x_k,0) \in \cu$ is a ramification point, i.e.\ $x_k \in X$, and
   \item $P_{\infty} \in \cu$ is an infinite point.
  \end{itemize}
  Finally, the resulting vector integral has to reduced modulo the period lattice, which is covered in \S \ref{m-subsec:lat_red}. We will give the 
  image of the Abel-Jacobi map in $\R^{2g} / \Z^{2g}$.
  
  Assume for this section that we already computed a period matrix (and all related data) following the Strategy from Section \ref{m-sec:strat_pm}. 
  
  Typically, we choose as basepoint the ramification point $P_0 = (x_0,0)$, where $x_0 \in X$ is the root of the spanning tree $T = (X,E)$.

  
   
  \subsection{Between ramification points}\label{subsec:ajm_bp}

  Consider a path $(x_0=x_{k_0},x_{k_1},\dots,x_{k_{n-1}},x_{k_t}=x_k)$ in the spanning tree connecting $x_0$ and $x_k$. Then we have that
  \begin{align}
    \int_{P_0}^{P_k} \bar\w = \sum_{j = 0}^{t-1}  \int_{P_{k_j}}^{P_{k_{j+1}}} \bar\w.
  \end{align}
  Denote $a = x_{k_j}, b = x_{k_{j+1}} \in X$. From \S \ref{m-subsec:cycles_homo} we know that for $(a,b) \in E$ a smooth path between $P_a=(a,0)$ and $P_b=(b,0)$ is given by
  \begin{align*}
   \gamma_{[a,b]}^{(0)} = \{  (x,\yab(x))  \mid  x \in [a,b]  \}.
  \end{align*}
  Fix a differential $\w_{i,j} \in \W$. Then, according to the proof of Theorem \ref{m-thm:periods} the corresponding integral is just an elementary integral
  \begin{align}
   \int_{\gamma_{[a,b]}^{(0)}} \w_{i,j}  & =  C_{i,j}  \int_{-1}^1 \frac{\varphi_{i,j}(u)}{(1-u^2)^{\frac{j}{m}}} \du,
  \end{align}
  which has already been evaluated during the period matrix computation.


  \subsection{Reaching non-ramification points}\label{subsec:ajm_finite}

   Let $P = (x_P,y_P)$ be a finite point on $\cu$ and $a \in X$ such that $X\cap]a,x_P]=\varnothing$. As in \S \ref{m-subsec:roots_branches} we define a locally analytic branch of $\cu$ via the functions
  \begin{align}\label{def:yax}
  \begin{split}
   \yaxp(x)  & =  \left(\frac{x_P-a}{2}\right)^{\frac{d}{m}}  (1 + u(x))^{\frac{j}{m}}  \ytaxp(u(x)), \quad \text{with} \\
   \ytaxp(u(x))  & =  \prod_{x_k \ne a} (u(x) - u(x_k))\mr \quad \text{and} \\
   u(x) & =  u_{a,x_P}(x) = \frac{2x-x_P-a}{x_P-a}.
   \end{split}
  \end{align}
  Now $\yaxp(x)$ is a branch of $\cu$ that is holomorphic in a neighbourhood of $]a,x_P]$.

  Define a path
  \begin{align*}
   \gamma_{a,x_P} = \{  (x,\yaxp(x))  \mid  x \in [a,x_P]  \}.
  \end{align*}
  Now, we can compute a \emph{shifting number} at the endpoint $x=x_P$
  \begin{align*}
   \zeta^s & = \frac{y_P}{\yaxp(x_P)} = \frac{y_P}{\left(\frac{x_P-a}{2}\right)^{\frac{d}{m}} 2\mr \ytaxp(x_P)} \\
   \Rightarrow s & = \frac{m}{2\pi} \arg\left(  \frac{y_P}{ (x_P-a)^{\frac{d}{m}} \ytaxp(x_P)} \right)
  \end{align*}
  and define a smooth path between $P_a$ and $P$ on $\cu$ by
  \begin{align*}
   \gamma_{P_a,P} = \{  (x, \zeta^s \yaxp(x))  \mid  x \in [a,x_P]  \}.
  \end{align*}



  \begin{thm}\label{thm:ajm_finite_int}
  Let $\w_{i,j} \in \WM$. With notation as above we have
 \begin{align}\label{eq:aj_int}
  \begin{split}
       \int_{P_a}^P \w_{i,j} & = \left(\frac{x_P-a}{2}\right)^{i-\frac{dj}{m}} \zeta^{-js} \int_{-1}^1
      \frac{\left(u + \frac{x_P+a}{x_P-a}\right)^{i-1}}{(1+u)^{\frac{j}{m}} \ytaxp(u)^j}  \du.
%      \ytaxp(u) & = \prod_{x_k \ne a} ( u - u_{a,x_P}(x_k) )\mr, \\
%      s & = \frac{m}{2\pi}  \arg \left( y_P \cdot (x_P-a)^{-\frac{d}{m}} \right).
   \end{split}
   \end{align}
  \end{thm}
  \begin{proof}
    \begin{align}
    \begin{split}
     \int_{P_a}^P \w_{i,j}  & =  \int_{\gamma_{P,a_P}} \frac{x^{i-1}}{y^j}  \dx  =  \zeta^{-sj} \int_a^{x_P} \frac{x^{i-1}}{\yaxp(x)^j}  \dx \\  & = 
     \zeta^{-sj} \left(\frac{x_P-a}{2}\right)^{i-1-\frac{dj}{m}}
     \int_a^{x_P} \frac{\left(u(x)+\frac{x_P+a}{x_P-a}\right)^{i-1}}{(1+u(x))^{\frac{j}{m}}\ytaxp(u(x))^j}  \dx \\
     & =  \zeta^{-sj} \left(\frac{x_P-a}{2}\right)^{i-\frac{dj}{m}}
     \int_{-1}^{1} \frac{\left(u+\frac{x_P+a}{x_P-a}\right)^{i-1}}{(1+u)^{\frac{j}{m}}\ytaxp(u)^j}  \du \\
    \end{split}
  \end{align}
  \end{proof}

  \subsection{Infinite points}\label{subsec:ajm_infty}

  Recall from \S \ref{m-subsec:sedef} that there are $\delta = \gcd(m,n)$ points above infinity $\mathcal{P}
  = \{ P_{\infty}^{(1)},\dots, P_{\infty}^{(\delta)} \}$ on our projective curve $\cu$.
  
  Suppose we want to integrate from $P_0$ to $P_{\infty} \in \mathcal{P}$, which is equivalent to computing the Abel-Jacobi map of the divisor
  $D_{\infty} = P_{\infty} - P_0$.
%   \begin{equation*}
%     D_{\infty} := \AJ([P_{\infty} - P_0]) \equiv \int_{P_0}^{P_{\infty}} \bar\w \mod \Lambda.
%   \end{equation*}
  Our strategy is to find a principal divisor $D \in \Prin(\cu)$ such that $\supp(D) \cap \mathcal{P} = \{ P_{\infty} \}$
  and $v_{P_{\infty}}(D) = \pm 1$. Then, by definition of the Abel-Jacobi map
  \begin{equation}
  \AJ([D_0]) \equiv \AJ([D_{\infty}]) \equiv \int_{P_0}^{P_{\infty}} \bar\w \mod \Lambda
  \end{equation}
  where $D_0 = D_{\infty} \mp D$ and $\supp(D_0) \cap \mathcal{P} = \varnothing$.
  Since there is no principal divisor who does the job unconditionally, we have to consider a few different cases.
  
  In the following we will use the Bézout identity
  \begin{equation}
    \mu m + \nu n = \delta,
  \end{equation}
  where $-\mu,\nu \in \Z_{\ge 0}$.
  
  
  \subsubsection{Coprime degrees}
  
  For $\delta = 1$ there is only one infinite point $\mathcal{P} = \{ P_{\infty} \}$ and
  we can easily compute  $\AJ( [D_{\infty}] )$ by adding a suitable
  principal divisor $D$
    \begin{align}
     \begin{split}
      &\div(y^{\nu}) ) =  \nu \sum_{k = 1}^n P_k - \nu n P_{\infty},\\
      &\div((x-x_0)^{\mu})  =  \mu m P_0 - \mu m P_{\infty} ,\\
      D  =  & \div(y^{\nu}(x-x_0)^{\mu})  = \nu \sum_{k = 1}^n P_k + \mu m P_0 - P_{\infty}.
     \end{split}
    \end{align}
    We immediately obtain
    \begin{align}
     \AJ( [D_{\infty}] )  & \equiv  \AJ([D_{\infty} + D])  =  \AJ( [ \nu \sum_{k = 1}^n P_k + (\mu m - 1) P_0 ]  ) \\  & \equiv  \AJ( [ \nu \sum_{k = 1}^d P_k ] )
      \equiv   \nu \sum_{k=1}^n \int_{P_0}^{P_k} \bar\w \mod \Lambda
    \end{align}
    and conclude that $\AJ( [D_{\infty}] )$ can be expressed in terms of integrals between
    ramification points (see \S \ref{m-subsec:ajm_bp}).
   
 \subsubsection{Non-coprime degrees}

  For $\delta > 1$ the problem becomes a lot harder. First we need a way to distinguish the infinite points in $\mathcal{P}
  = \{ P_{\infty}^{(1)},\dots, P_{\infty}^{(\delta)} \}$ and second they are singular points
  on the homogenization of our affine model $\caff$
  whenever $m \ne n,n\pm1$.
  
  As shown in \cite[\S 1]{CT1996}, we obtain a different affine model of $\cu$ that is non-singular at $\mathcal{P}$
  in the following way:
  
  Denoting $M = \frac{m}\delta$ and $N = \frac{n}\delta$, we consider the birational transformation
  \begin{equation}
   (x,y) = \Phi(r,t) = \left(\frac{1}{r^{\nu}t^M},\frac{r^{\mu}}{t^N}\right)
  \end{equation}
  which results in an affine model
  \begin{equation}
   \cafft : r^{\delta} = \prod_{k=1}^n (1-x_kr^{\nu}t^M).
  \end{equation}
  The inverse transformation is given by
  \begin{equation}
   (r,t) = \Phi^{-1}(x,y) = \left(\frac{y^M}{x^N},\frac{1}{x^{\mu}y^{\nu}}\right).
  \end{equation}
  Under this transformation the infinite points in $\mathcal{P}$ are mapped to finite points that have the coordinates
  \begin{equation}
   (r,t) = (\zeta_{\delta}^s,0) \quad s= 1,\dots,\delta
  \end{equation}
  where $\zeta_{\delta} = e^{\frac{2\pi i }{\delta}}$. 
  Hence, we can identify the infinite points in $\mathcal{P} \subset \cu$ via
   \begin{equation}
      P_{\infty}^{(s)} = \Phi^{-1}(\zeta_{\delta}^s,0).
   \end{equation}
   
   
   Suppose we want to compute the Abel-Jacobi map of $D_{\infty}^{(s)} = P_{\infty}^{(s)} - P_0$ for $s \in \{1,\dots,\delta\}$.
   Again following our strategy,
   this time on $\cafft$, we look at the divisor of the vertical line through $(\zeta_{\delta}^s,0)$.
   \begin{equation}
      D_1 = \div(r - \zeta_{\delta}^s) = \sum_{i = 1}^{\deg(g)} (\zeta_{\delta}^s,t_i) - N D_1'
   \end{equation}
      where the $t_i$ are the zeros of $g(t) = \prod_{k=1}^n (1-x_k\zeta_{\delta}^{s\nu}t^M) - 1 \in \C[t]$ and
    \begin{equation}
       D_1' = \begin{cases}
             \sum_{l=1}^m \Phi^{-1}(0,\zeta_m^l \sqrt[m]{f(0)}), \quad \text{if} \; 0 \not\in X,\\
             (m-M) \Phi^{-1}(0,0), \hfill \text{if} \; 0 \in X.
            \end{cases}
    \end{equation}
    Now, $\Phi(D_1) := \div \left( \frac{y^M}{x^N} - \zeta_{\delta}^s \right)$ is a principal divisor on $\caff$ and by
    construction we have $v_{P_{\infty}^{(s)}}(\Phi(D_1)) \ge 1$.
    
  
    If $v_{P_{\infty}^{(s)}}(\Phi(D_1)) = 1$ we are done: assuming $t_1 = 0$, we obtain
    \begin{align}
      \AJ( [D_{\infty}^{(s)}] )  & \equiv  \AJ([D_{\infty}^{(s)} - \Phi(D_1)])  
      \equiv  -\AJ( [\sum_{i = 2}^{\deg(g)} \Phi(\zeta_{\delta}^s,t_i) + N \Phi(D_1') ] )
%      \\ & \equiv  -\sum_{i = 2}^{\deg(g)}  \AJ( \Phi(\zeta_{\delta}^s,t_i)) + N \AJ( \Phi(D_1') )
       \mod \Lambda,
    \end{align}
    which involves only finite points on $\caff$ and can thus be computed using Theorem \ref{m-thm:ajm_finite_int}.
    
    \bigskip
    
    In the case where $v_{P_{\infty}^{(s)}}(\Phi(D_1))$ is greater than $1$, the vertical line is tangent to the curve $\cafft$ 
    at $(\zeta_{\delta}^s,0)$ and cannot be used.
    Consequently, we must find another function. The obvious choice here is the line $r - t -  \zeta_{\delta}^s$, which
    is now guaranteed to have a simple intersection with $\cafft$ 
    at $(\zeta_{\delta}^s,0)$.

    The corresponding principal divisor is given by
     \begin{equation}
      D_2 = \div(r - t - \zeta_{\delta}^s) = \sum_{i = 1}^{\deg(g)} (t_i+\zeta_{\delta}^s,t_i) - b \sum_{k=1}^n
      \Phi^{-1}(x_k,0)- N D_2'
   \end{equation}
      where the $t_i$ are the zeros of $g(t) = \prod_{k=1}^n (1-x_k(t+\zeta_{\delta})^{s\nu}t^M) - 1 \in \C[t]$ and
    \begin{equation}
       D_2' = \begin{cases}
             \sum_{l=1}^m \Phi^{-1}(0,\zeta_m^l \sqrt[m]{f(0)}), \quad \text{if} \; 0 \not\in X,\\
             (m-\frac{M+\nu}{N}) \Phi^{-1}(0,0), \hfill \text{if} \; 0 \in X.
            \end{cases}
    \end{equation}
 
    Now, $\Phi(D_2) := \div \left( \frac{y^M}{x^N} - \frac{1}{x^{\mu}y^{\nu}} - \zeta_{\delta}^s \right)$
    is a principal divisor such that
    $v_{P_{\infty}^{(s)}}(\Phi(D_2))$ is equal to $1$. Hence, with $t_1 = 0$ we have
     \begin{align}
      \AJ( [D_{\infty}^{(s)}] )  & \equiv  \AJ([D_{\infty}^{(s)} - \Phi(D_2)])  \\
      & \equiv  -\AJ( [\sum_{i = 2}^{\deg(g)} \Phi(\zeta_{\delta}^s,t_i) + b \sum_{k=1}^n
      (x_k,0) + N \Phi(D_2') ] )
%      \\ & \equiv  -\sum_{i = 2}^{\deg(g)}  \AJ( \Phi(\zeta_{\delta}^s,t_i)) + N \AJ( \Phi(D_1') )
       \mod \Lambda.
    \end{align}
    
  \subsection{Reduction modulo period lattice}\label{subsec:lat_red}

    In order for the Abel-Jacobi map to be well defined we have to reduce modulo the period lattice $\Lambda = 
  \Omega\Z^{2g}$, where $\Omega = (\OA, \OB)$ is the big period matrix, computed as explained in
  Section \ref{m-sec:strat_pm}.
   
   For reasons that will be adressed later in this paragraph
   we choose to
   represent the image of the Abel-Jacobi map in 
    $\R^{2g} / \Z^{2g}$.
    
   Let $v = \int_P^Q \bar\w \in \C^g$ be an vector integral obtained by integration the holomorphic differentials in
   $\W$.
   We identify the complex and real vector space via the bijection
   \begin{equation}
    \iota:\C^g \rightarrow \R^{2g},v = (v_1,\dots,v_g)^T \mapsto (\Re(v_1),\dots,\Re(v_g),\Im(v_1),\dots,\Im(v_g))^T.
   \end{equation}
    Applying $\iota$ to the columns of $\Omega$ yields the invertible real matrix
   \begin{equation}
    R = 
   \begin{pmatrix}
     \Re(\OA) & \Re(\OB) \\
     \Im(\OA) & \Im(\OB)
    \end{pmatrix} \in \R^{2g \times 2g}.
   \end{equation}
   Now, reduction of $v$ modulo $\Lambda$ corresponds to taking the fractional part of $\iota(v)R^{-1}$
   \begin{equation}
    v \bmod \Lambda \leftrightarrow \lfloor \iota(v)R^{-1} \rceil.
   \end{equation}
   This representation has the following advantages:
   \begin{rmk}[Image of Abel-Jacobi map] \
    \begin{itemize}
     \item[$\bullet$] Operations on the Jacobian variety $\Jac(\cu)$ correspond to operations in $\R^{2g}/\Z^{2g}$.
     \item[$\bullet$] Torsion divisors under $\AJ$ are mapped to vectors of rational numbers with 
     denominator dividing $m$, i.e. 
     \begin{equation}
      \AJ([P_k - P_j]) \equiv \int_{P_0}^{P_k} \bar\w \bmod \Lambda \leftrightarrow \lfloor \iota\left(\int_{P_0}^{P_k} \bar\w\right)
      R^{-1} \rceil \in \frac{1}m\Z^{2g}
     \end{equation}
     for $k,j \in \{ 1,\dots,n \}$.
    \end{itemize}

    
   \end{rmk}

   
   
   
    

\biblio
\end{document}
