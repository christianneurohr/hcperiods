\documentclass[main.tex]{subfiles}

\begin{document}

  \section{Strategy for the period matrix}\label{sec:strat_pm}

  In this section we present our strategy to obtain period matrices $\OC, \OA, \OB$ and $\tau$ as defined in
  \S \ref{m-subsec:bases_matrices}. Although this paper is not restricted to the
  case $\gcd(m,n) = 1$, we will briefly assume it in this paragraph to simplify notation.

  The main ingredients were already described in
  Section \ref{m-sec:se_curves}: we integrate the holomorphic differentials in $\W$ (\S \ref{m-subsec:diff_forms})
  over the cycles in $\Gamma$ (\S \ref{m-subsec:cycles_homo}) using numerical integration (\S \ref{m-sec:de_int}), which results in a period matrix (\S \ref{m-subsec:comp_of_periods})
  \begin{equation*}
      \label{eq:OC}
    \OC = \left( \int_{\gamma} \w \right)_{\substack{\w \in \W, \\ \gamma \in \Gamma}} \in \C^{g \times 2g}.
  \end{equation*}
  The matrices $\OA$ and $\OB$ require a symplectic basis of $\homo$.
  So, we compute the intersection pairing on $\Gamma$, as explained in Section \ref{m-sec:intersections}, which results in a
  intersection matrix $K_{\Gamma} \in \Z^{2g \times 2g}$.
  After computing a symplectic base change $S \in \GL(\Z,2g)$ for $K_{\Gamma}$ (\S \ref{m-subsec:symp_basis}), we obtain a big period matrix
  \begin{equation*}
      \label{eq:OAOB}
      (\OA,\OB) = \OC S,
  \end{equation*}
   and finally a small period matrix in the Siegel upper half-space
  \begin{equation*}
      \label{eq:tau}
   \tau = \OA^{-1} \OB \in \mathfrak{H}_g.
  \end{equation*}

  \bigskip

  \subsection{Periods of elementary cycles}\label{subsec:comp_of_periods}

  The following theorem provides a formula for computing the periods of the curve.
  It relates integration of differential forms on the curve to numerical integration in $\C$.

  Note that the statement is true for all differentials in $\WM$, not just the holomorphic ones.
  We continue to use the notation from Section \ref{m-sec:se_curves}.

  \begin{thm}\label{thm:periods}
   Let $\gamma_e^{(l)} \in \Gamma$ be a shift of an elementary cycle corresponding
   to an edge $e = (a,b) \in E$. Then, for all differentials $\w_{i,j} \in \WM$, we have
   \begin{equation}\label{eq:periods}
      \int_{\gamma_e^{(l)}} \w_{i,j}  =  \zeta^{-lj} (1-\zeta^{-j}) C_{a,b}^{-j} \left(\frac{b-a}{2}\right)^i \int_{-1}^1 \frac{\varphi_{i,j}(u)}{(1-u^2)^{\frac{j}{m}}}  \du
   \end{equation}
   where
   \begin{equation*}
    \varphi_{i,j}  = \left(u+\frac{b+a}{b-a}\right)^{i-1} \ytab(u)^{-j}
   \end{equation*}
   is holomorphic in a neighbourhood $\epsilon_{a,b}$ of $[-1,1]$.
  \end{thm}
  \begin{proof}
    By the definition in \eqref{m-eq:def_cyabl} we can write $\gamma_e^{(l)} = \gamma_{[a,b]}^{(l)} \cup \gamma_{[b,a]}^{(l+1)}$. Hence we split up the integral and compute
    \begin{align*}\label{eq:thm_periods_2}
     \int_{\gamma_{[a,b]}^{(l)}} \w_{i,j}  & =  \int_{\gamma_{[a,b]}^{(l)}} \frac{x^{i-1}}{y^j}  \dx  =  \zeta^{-lj} \int_a^b \frac{x^{i-1}}{\yab(x)^j}  \dx \\  & =
     \zeta^{-lj} C_{a,b}^{-j}   \int_a^b \frac{x^{i-1}}{\ytab(\uab(x))^j (1-\uab(x)^2)^{\frac{j}{m}}}  \dx
     \intertext{
  Applying the transformation $x \mapsto \xab(u)$ introduces the derivative $\dx = \left(\frac{b-a}{2}\right) \du$.
  }
   & = \zeta^{-lj} C_{a,b}^{-j} \left(\frac{b-a}{2}\right) \int_{-1}^1 \frac{\uab(u)^{i-1}}{\ytab(u)^j (1-u^2)^{\frac{j}{m}}}  \du \\ & =
    \zeta^{-lj} C_{a,b}^{-j} \left(\frac{b-a}{2}\right)^i \int_{-1}^1 \frac{\left(u+\frac{b+a}{b-a}\right)^{i-1}}{\ytab(u)^j (1-u^2)^{\frac{j}{m}}}  \du
  \end{align*}
  Similarly, we obtain
  \begin{equation*}
        \int_{\gamma_{[b,a]}^{(l+1)}} w_{i,j}  =  -\zeta^{-j} \int_{\gamma_{[a,b]}^{(l)}} w_{i,j}.
  \end{equation*}
  By Proposition \ref{m-prop:yab} $\ytab$ is holomorphic and has no zero on $\epsilon_{a,b}$, therefore \\ $\varphi_{i,j}  = \left(u+\frac{b+a}{b-a}\right)^{i-1} \ytab(u)^{-j}$
  is holomorphic on $\epsilon_{a,b}$.
  \end{proof}

   \subsection{Numerical integration}\label{subsec:numerical_integration}

   In order to compute a period matrix $\OC$ the only integrals that have to
   be numerically evaluated are
   the \emph{elementary integrals}
   \begin{equation}\label{eq:elem_ints}
       \int_{-1}^1 \frac{\varphi_{i,j}(u)}{(1-u^2)^{\frac{j}{m}}} \du
   \end{equation}
   for all $\w_{i,j} \in \W$ and $e \in E$. By Theorem \ref{m-thm:periods}, all the periods in $\OC$ are
   then obtained by multiplication of elementary integrals with constants.

 As explained in \S \ref{m-subsec:real_mult}, the actual computations will be done on integrals of the form
\begin{equation}
    \label{eq:elem_num_int}
    I_{a,b}(i,j) = \int_{-1}^1\frac{u^{i-1}\du}{(1-u^2)^{\frac jm}\ytab(u)^j}
\end{equation}
(that is, replacing $(u+\frac{b+a}{b-a})^{i-1}$ by $u^{i-1}$ in the numerator of $\phi_{i,j}$)
the value of elementary integrals being recovered by the polynomial shift
\begin{equation}
    \label{eq:polshift}
    \int_{-1}^1\frac{\varphi_{i,j}(u)}{(1-u^2)^{\frac jm}}\du
    = \sum_{l=0}^{i-1} {i-1 \choose l} \left(\frac{b+a}{b-a}\right)^{i-1-l} I_{a,b}(l,j).
\end{equation}

The rigorous numerical evaluation of \eqref{m-eq:elem_num_int} is
adressed in Section~\ref{m-sec:numerical_integration}: for any edge $(a,b)$,
Theorem~\ref{m-thm:de_int} and \ref{m-thm:gc_int} provide
an explicit scheme allowing to attain any prescribed precision.

  \subsection{Minimal spanning tree}\label{subsec:spanning_tree}


  From the a priori analysis of all numerical integrals $I_{a,b}$ along
  the interval $[a,b]$, we choose an optimal set of edges forming a spanning tree as follows:
  \begin{itemize}
      \item
   Consider the complete graph on the set of finite branch points $G' = (X,E')$ where
   $E' = \{  (a,b)  \mid  a,b \in X \}$.
      \item
   Each edge $e = (a,b) \in E'$ gets assigned a capacity $r_e$ that indicates
   the cost of numerical integration along the interval $[a,b]$.
   \item
   Apply a standard `maximal-flow' algorithm from graph theory, based on a greedy approach,
   that results in a spanning tree $G = (X,E)$, where $E \subset E'$ contains the $n-1$ best edges
   for integration that connect all vertices without producing cycles.
  \end{itemize}

   Notice that the integration process is most favourable
   between branch points that are far away from the others
   (this notion is made explicit in Section \ref{m-sec:numerical_integration}).

  \subsection{Symplectic basis}\label{subsec:symp_basis}

  By definition, a big period matrix $(\OA,\OB)$ requires integration along
  a symplectic basis of $\homo$. In \S \ref{m-subsec:cycles_homo} we gave
  a generating set $\Gamma$ for $\homo$, namely
  \begin{equation*}
    \Gamma = \left\{  \gamma_{e}^{(l)}  \mid  0 \le l < m-1,  e \in E  \right\}
  \end{equation*}
  where $E$ is the spanning tree chosen above. This generating set is
  in general not a (symplectic) basis.

  We resolve this by computing the intersection pairing on $\Gamma$, that
  is all intersections
  $\gamma_{e}^{(k)} \circ \gamma_{f}^{(l)}\in\{ 0,\pm 1\}$ for $e,f\in E$ and $k,l\in \set{0,\dots m-1}$,
  as explained in Section \ref{m-sec:intersections}.

  The resulting intersection matrix $K_{\Gamma}$
   is a skew-symmetric matrix of dimension \\ $(n-1)(m-1)$
   and has rank $2g$.

   Hence, we can apply an algorithm, based on
    \cite[Theorem 18]{KB2002}, that outputs a symplectic basis
   for $K_{\Gamma}$ over $\Z$, i.e. a unimodular matrix base change matrix $S$ such that
  $$S^T K_{\Gamma}  S = J, \quad \text{where} \quad J = \begin{pmatrix} 0 & I_g & 0 \\ -I_g & 0 & 0 \\ 0 & 0 & 0_{\delta-1} \end{pmatrix}.$$

  The linear combinations of periods given by the first $2g$ columns of $\OC S$ then correspond to a symplectic homology basis
  \begin{equation*}\label{eq:symp_basis}
   (\OA,\OB, 0_{\delta-1}) = \OC S
  \end{equation*}
  whereas the last $\delta-1$ columns are zero and can be ignored, as they correspond to the dependent cycles
  in $\Gamma$ and contribute nothing.

\biblio
\end{document}
