\documentclass[main.tex]{subfiles}

\begin{document}

  \section{Strategy for the period matrix}\label{sec:strat_pm}

  In this section we present our strategy to obtain period matrices $\OC, \OA, \OB$ and $\tau$ as defined in \S \ref{m-subsec:bases_matrices}. Although our algorithm is no way restricted to the
  case $\delta = 1$, we will briefly assume it to simplify notation.
  
  The main ingredients were already described in
  Section \ref{m-sec:se_curves}: we integrate the holomorphic differentials in $\W$ (\S \ref{m-subsec:diff_forms})
  over the cycles in $\Gamma$ (\S \ref{m-subsec:cycles_homo}) using numerical integration (\S \ref{m-sec:de_int}), which results in a period matrix (\S \ref{m-subsec:comp_of_periods})
  \begin{equation}
      \label{eq:OC}
    \OC = \left( \int_{\gamma} \w \right)_{\substack{\w \in \W, \\ \gamma \in \Gamma}} \in \C^{g \times 2g}.
  \end{equation}
  The matrices $\OA$ and $\OB$ require a symplectic basis of $\homo$.
  So, we compute the intersection pairing on $\Gamma$, as explained in Section \ref{m-sec:intersections}, which results in a
  intersection matrix $K_{\Gamma} \in \Z^{2g \times 2g}$.
  After computing a symplectic basis $S \in \GL(\Z,2g)$ for $K_{\Gamma}$ (\S \ref{m-subsec:symp_basis}), we obtain a big period matrix
  \begin{equation}
      \label{eq:OAOB}
      (\OA,\OB) = \OC S,
  \end{equation}
   and finally a small period matrix in the Siegel upper half-space
  \begin{equation}
      \label{eq:tau}
   \tau = \OA^{-1} \OB \in \mathfrak{H}_g.
  \end{equation}

  \bigskip

  \subsection{Computation of periods}\label{subsec:comp_of_periods}

  The following theorem provides a formula for computing the periods of $\cu$. It relates integration of differential forms on the curve to numerical integration in $\C$.
  Although we only need to integrate holomorphic
  differentials, the statement is true for all differentials in $\WM$.

  We continue to use the notation from Section \ref{m-sec:se_curves}.

  \begin{thm}\label{thm:periods}
   Let $\gamma_e^{(l)} \in \Gamma$ be a shift of an elementary cycle corresponding
   to an edge $e = (a,b) \in E$. Then, for all differentials $w_{i,j} \in \WM$, we have
   \begin{equation}\label{eq:periods} 
      \int_{\gamma_e^{(l)}} \w_{i,j}  =  \zeta^{-lj} (1-\zeta^{-j}) C_{a,b}^{-j} \left(\frac{b-a}{2}\right)^i \int_{-1}^1 \frac{\varphi_{i,j}(u)}{(1-u^2)^{\frac{j}{m}}}  \du
   \end{equation}
   where
   \begin{equation}
    \varphi_{i,j}  = \left(u+\frac{b+a}{b-a}\right)^{i-1} \ytab(u)^{-j}
   \end{equation}
   is holomorphic in a neighbourhood $\epsilon_{a,b}$ of $[-1,1]$.
  \end{thm}
  \begin{proof}
    By the definition in \eqref{m-eq:def_cyabl} we can write $\gamma_e^{(l)} = \gamma_{[a,b]}^{(l)} \cup \gamma_{[b,a]}^{(l+1)}$. Hence we split up the integral and compute
    \begin{align}\label{eq:thm_periods_2}
     \int_{\gamma_{[a,b]}^{(l)}} \w_{i,j}  & =  \int_{\gamma_{[a,b]}^{(l)}} \frac{x^{i-1}}{y^j}  \dx  =  \zeta^{-lj} \int_a^b \frac{x^{i-1}}{\yab(x)^j}  \dx \\  & =
     \zeta^{-lj} C_{a,b}^{-j}   \int_a^b \frac{x^{i-1}}{\ytab(\uab^{-1}(x))^j (1-\uab^{-1}(x)^2)^{\frac{j}{m}}}  \dx
  \end{align}
  Applying the transformation $x \mapsto \uab(u)$ introduces the derivative $\dx = \left(\frac{b-a}{2}\right) \du$.
  \begin{align}
   & = \zeta^{-lj} C_{a,b}^{-j} \left(\frac{b-a}{2}\right) \int_{-1}^1 \frac{\uab(u)^{i-1}}{\ytab(u)^j (1-u^2)^{\frac{j}{m}}}  \du \\ & = 
    \zeta^{-lj} C_{a,b}^{-j} \left(\frac{b-a}{2}\right)^i \int_{-1}^1 \frac{\left(u+\frac{b+a}{b-a}\right)^{i-1}}{\ytab(u)^j (1-u^2)^{\frac{j}{m}}}  \du
  \end{align}
  Similarly, we obtain
  \begin{align}
        \int_{\gamma_{[b,a]}^{(l+1)}} w_{i,j}  & =  -\zeta^{-j} \int_{\gamma_{[a,b]}^{(l)}} w_{i,j}.
  \end{align}
  By Proposition \ref{m-prop:yab} $\ytab$ is holomorphic and has no zeros on $\epsilon_{a,b}$, therefore \\ $\varphi_{i,j}  = \left(u+\frac{b+a}{b-a}\right)^{i-1} \ytab(u)^{-j}$
  is holomorphic on $\epsilon_{a,b}$.
  \end{proof}

  \begin{rmk}\label{rmk:elem_ints}
   From Theorem \ref{m-thm:periods} it follows that we can obtain a period matrix by computing the \emph{elementary integrals}
   \begin{equation}\label{eq:elem_ints}
       \int_{-1}^1 \frac{\varphi_{i,j}(u)}{(1-u^2)^{\frac{j}{m}}}  \du
   \end{equation}
   for all $\w_{i,j} \in \W$ and $e \in E$. All other integrals are obtained by multiplication with constants.
   \end{rmk}
 
  
  
  \subsection{Symplectic basis}\label{subsec:symp_basis}

  For a period matrix $\tau$, we need to integrate over a symplectic basis of $\homo$. In Section \ref{m-subsec:cycles_homo}, we chose a generating set $\Gamma$ for $\homo$, that is in general not
  a (symplectic) basis. We ressolve this by computing the intersection pairing on $\Gamma$, as explained in Section \ref{m-sec:intersections}. 
  
  The resulting intersection matrix $K_{\Gamma}$
   is a skew-symmetric matrix of dimension \\ $(n-1)(m-1)$ with entries in $\{ 0,\pm 1\}$ and has rank $2g$. Hence, we can apply an algorithm, based on
    \cite[Theorem 18]{KB2002}, that computes a symplectic basis
   for $K_{\Gamma}$ over $\Z$, i.e. a unimodular matrix $S$ such that
  $$S^T K_{\Gamma}  S = J, \quad \text{where} \quad J = \begin{pmatrix} 0 & I_g & 0 \\ -I_g & 0 & 0 \\ 0 & 0 & 0_{\delta-1} \end{pmatrix}.$$
  The linear combinations of cycles given by $\OC S$ then yield a symplectic homology basis
  \begin{equation}\label{eq:symp_basis}
   (\OA,\OB, 0_{\delta-1}) = \OC S.
  \end{equation}
  Note that the last $\delta-1$ columns of $\OC S$ are zero and can be ignored, as they correspond to the dependent cycles
  in $\Gamma$ and contribute nothing.


\biblio
\end{document}
