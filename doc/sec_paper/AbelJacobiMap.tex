\documentclass[main.tex]{subfiles}

\begin{document}

  \section{The Abel-Jacobi map}

  \subsection{Definition}

  Let $\cu$ be a smooth irreducible curve of genus $g$. Its space
  of holomorphic differentials $\hd$ has dimension $g$, let us fix
  a basis $ω_1,\dots ω_g$.

  For any two points $P_0,P$ we can 
  consider the integral $\int_{P_0}^Pω_j\in\C^g$ but its value
  depends on the chosen path from $P_0$ to $P$.

  In fact the integral is defined up to homology on the path,
  so we consider the {\em period lattice} of $\cu$
  \begin{equation}
      \Lambda = \set{\int_γ ω_j, γ\in\homo} \subset\C^g
  \end{equation}
  where $\homo \simeq\Z^{2g}$ is the first homology group
  of the curve.

  Now the integral
  \begin{equation}
      P \mapsto \int_{P_0}^P ω_j \in \C^g/\Lambda
  \end{equation}
  is well defined, and the definition can be extended
  by linearity to the group $D^0(\cu)$ of zero degree divisors.

  The Abel-Jacobi theorem states that one obtains a
  surjective map %to the torus $\C^g/\Lambda$ and
  whose kernel
  is formed by divisors of functions, so that the integration
  provides an explicit isomorphism
  
  \begin{equation}\label{eq:ajm_def}
      \AJ:\left\{\begin{array}{ccc}
              \Jac = \div^0/P^0 &\To &\C^g/\Lambda \\
              \sum_k [Q_k-P_k] &\mapsto & \sum_k \int_{P_k}^{Q_k} \omega_j \mod \Lambda
  \end{array}\right.
  \end{equation}
  between the Jacobian variety and the complex torus. \\
  \todo (CN) $D^0 = \div^0 = \div_0(\cu)$? $P^0$ = principal divisors? unclear.
  

  \subsection{Explicit basis and standard matrices}\label{subsec:bases_matrices}

  %There exist a skew-symmetric intersection product on paths
  Let us choose a symplectic basis of $\homo$, that is two
  families of loops $α_i$, $β_j$ for $1\leq i,j\leq g$ such that
  %the intersections
  the intersections satisfy 
  \begin{equation}
      %α_i\cdot β_j = \delta_{i,j}
      \left( \,\alpha_i \circ \beta_j \,\right) = \delta_{i,j}\,,
  \end{equation}
  the other intersections being all zero. 

  We define the period matrices
  \begin{equation}
      \OA = \left(\int_{α_i}ω_j\right)_{1\leq i,j\leq g}
      \text{ and }
      \OB = \left(\int_{β_i}ω_j\right)_{1\leq i,j\leq g}
  \end{equation}
  and call {\em big period matrix} the matrix \\
  \todo (CN) without comma? $(\OA \, \OB)$? $\in \C^{g \times 2g}$ or $\C^{2g \times g}$?
  \begin{equation}
      \Omega = (\OA,\OB)
  \end{equation}
  such that $\Lambda = \Omega\Z^{2g}$.

  If one takes as basis of differentials the dual basis of
  the loops $α_i$ the matrix becomes
  \begin{equation}
      \OA^{-1}\Omega = (I_g,\tau)
  \end{equation}
  where $\tau = \OA^{-1}\OB$ is in the Siegel space of
  symmetric matrices with positive definite imaginary part.

\biblio
\end{document}
