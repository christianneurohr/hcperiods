\documentclass[main.tex]{subfiles}

\begin{document}

  \section{The Abel-Jacobi map}\label{sec:ajm}

  We recall, without proof, the main objects we are interested in, and which
  will become completely explicit in the case of superelliptic curves.
  The exposition follows that of \cite[Section 2]{vanWam1998}.

  \subsection{Definition}

  Let $\cu$ be a smooth irreducible projective curve of genus $g>0$. Its space
  of holomorphic differentials $\hd$ has dimension $g$; let us fix
  a basis $ω_1,\dots ω_g$ and denote by $\bar\w$ the vector
  $(ω_1,\dots ω_g)$.

  For any two points $P,Q \in \cu$ we can
  consider the vector integral $\int_{P}^Q\bar\w\in\C^g$, whose value
  depends on the chosen path from $P$ to $Q$.

  In fact, the integral depends on the path up to homology,
  so we introduce the {\em period lattice} of $\cu$
  \begin{equation*}
      \Lambda = \set{\int_γ ω_j, γ\in\homo} \subset\C^g,
  \end{equation*}
  where $\homo \isom\Z^{2g}$ is the first homology group
  of the curve.

  Now the integral
  \begin{equation*}
      P,Q \mapsto \int_{P}^Q \bar\w \in \C^g/\Lambda
  \end{equation*}
  is well defined, and the definition can be extended
  by linearity to the group of
  degree zero divisors
  \begin{equation*}
      \Div^0(\cu)=\set{ \sum a_i P_i, a_i\in\Z, \sum a_i = 0}.
  \end{equation*}

  The Abel-Jacobi theorem states that one obtains a
  surjective map %to the torus $\C^g/\Lambda$ and
  whose kernel
  is formed by divisors of functions, so that the integration
  provides an explicit isomorphism
  \begin{equation*}\label{eq:ajm_def}
      \AJ:\left\{\begin{array}{ccc}
              \Jac(\cu) = \Div^0(\cu)/\Prin^0(\cu) &\To &\C^g/\Lambda \\
              \sum_i [Q_i-P_i] &\mapsto & \sum_k \int_{P_i}^{Q_i} \bar\w \mod \Lambda
  \end{array}\right.
  \end{equation*}
  between the Jacobian variety and the complex torus. \\


  \subsection{Explicit basis and standard matrices}\label{subsec:bases_matrices}

  Let us choose a symplectic basis of $\homo$, that is two
  families of cycles $α_i$, $β_j$ for $1\leq i,j\leq g$ such that
  the intersections satisfy
  \begin{equation*}
      %α_i\cdot β_j = \delta_{i,j}
      \left( \alpha_i \circ \beta_j \right) = \delta_{i,j},
  \end{equation*}
  the other intersections all being zero.

  We define the period matrices on those cycles
  \begin{equation*}
      \OA = \left(\int_{α_i}ω_j\right)_{1\leq i,j\leq g}
      \text{ and }
      \OB = \left(\int_{β_i}ω_j\right)_{1\leq i,j\leq g}
  \end{equation*}
  and call the concatenated matrix \\
  \begin{equation*}
      \Omega = (\OA, \OB) \in \C^{g\times 2g}
  \end{equation*}
  such that $\Lambda = \Omega\Z^{2g}$ a {\em big period matrix}. 

  If one takes as basis of differentials the dual basis of
  the cycles $α_i$, the matrix becomes
  \begin{equation*}
      \OA^{-1}\Omega = (I_g, \tau),
  \end{equation*}
  where $\tau = \OA^{-1}\OB \in \C^{g \times g}$, called  a {\em small period matrix}, is in the Siegel space
  $\mathcal{H}_g$ of symmetric matrices with positive definite imaginary part.
\biblio
\end{document}
