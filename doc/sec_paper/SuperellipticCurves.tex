\documentclass[main.tex]{subfiles}

\begin{document}

  \section{Superelliptic curves}

  \subsection{Definition \& properties}
    Content: Definition, affine model, branch points, genus, monic transform?, Riemann surface, pts at infinity
    
    \begin{defn}\label{def:SEDEF}
       A superelliptic curve over a field $K$ is a smooth cyclic branched covering of the projective line 
       $\cu \rightarrow \P^1_K$ of degree $m > 1$ such that $\text{char}(K)$ and $m$ are coprime.
   \end{defn}
   In this paper we consider superelliptic curves over $K = \C$ that have an affine model given by an equation of the form
    \begin{align}\label{eq:aff_model}
     \caff : \quad y^m = f(x) \;= \; c_f \cdot \prod_{k=1}^d (x-x_k)
    \end{align}
    where $f \in \C[x]$ is separable of degree $d \ge 3$.
    Without loss of generality we may assume $c_f = 1$\,. (If not, apply the transformation $(x,y) \mapsto (x,\sqrt[N]{c_f}y)$.)
    We denote by $X = \X$ the set of (finite) branch points all of which are totally ramified. By definition
  the local monodromy at each finite branch point is equal and can be represented by a cyclic permutation of length $m$.
  Let $\delta = \gcd(m,d)$. There are $\delta$ points above infinity $P_1^{(\infty)},\dots,P_{\delta}^{(\infty)}$. Therefore
  the point at infinity is a branch point for $\delta \ne m$. For more details see \todo ref.
  
  \subsection{Complex roots and branches of the curve}
    Content: - Complex N-th root, branches $\yab(x)$, analytic continuation, monodromy, sheets
  
  
%   For complex numbers $0 \ne z = r e^{i\theta} \in \C$ with $r \in \R_{\ge 0}$ we choose its argument $\arg(z) = \theta$ to be in $(-\pi,\pi]$. 
%   We fix a primitive $N$-th root of unity $\zeta = \zeta_N := e^{\frac{2\pi i}{N}}$ and its square root $\sqrt{\zeta} := e^{\frac{\pi i}{N}}$.
%   The set of $N$-th roots of $z$ is then given by $\{ \, \zeta^k r^{\frac{1}{N}} e^{i\frac{\theta}{N}} \, \mid \, k = 0,\dots,N-1 \, \}$, where
%   $r^{\frac{1}{N}}$ is the positve real $N$-th root of $r$\,.
%   Thus, we can see the taking $N$-th roots of a complex number as a multi-valued complex function with $N$ branches. 
%    
%   We define the $N$-th root function by
%     $$(\, \cdot \, )^{\frac{1}{N}} \, = \, \sqrt[N]{\cdot} \, : \, \C \rightarrow \C,\; z \mapsto \begin{cases}
%                                                        \, r^{\frac{1}{N}} \, e^{i\frac{\theta}{N}}, \quad \text{if} \; z \ne 0\,, \\
%                                                        \, 0, \quad \text{if} \; z = 0\,.
%                                                      \end{cases}$$
% 
%   This function has a branch cut along the negative real axis $]-\infty,0]$ and is holomorphic everywhere else.
%   
%   For $z \in \C$ we call $\sqrt[N]{z}$ as defined above the \textit{standard determination/principal value} of the $N$-th root of $z$.
  Working over the complex we encounter several multi-valued functions for which principal branches have to be specified. In order to avoid confusion we briefly discuss conventions here. 
  For $z \in C$ we choose its argument $\arg(z) \in (-\pi,\pi]$ and its $m$-th root such that $\arg \sqrt[m]{z} \in$ ...
  Closely related to a superelliptic curve over $\C$ is complex $m$-th root function.
  
  
  For $a,b \in X$ consider the affine transformation $x \mapsto \frac{b-a}{2}\left(u+\frac{b+a}{b-a}\right)$, which maps $[a,b]$ to $[-1,1]$ and will appear in the double-exponential integration. \abstand
  We define the following functions
  \begin{align}\label{def: yab}
  \begin{split}
   \yab(x) & = \left(\frac{b-a}{2}\right)^{\frac{d}{m}} \sqrt{\zeta} \; (1 - u(x)^2)\mr \; \ft_{a,b}(x)\,, \quad \text{with} \\
   \ft_{a,b}(x) & = \prod_{x_k \ne a,b} (u(x) - u(x_k))\mr \quad \text{and} \\
   u_{a,b}(x) & = u(x) = \frac{2x-b-a}{b-a}\,.
   \end{split}
  \end{align}
  where all roots are given by the principal value. Under the condition that $x_k \not\in [a,b]$, the function $u(x)-u(x_k)$ does not cross $]0,-\infty[$
  for $x \in [a,b]$ and $x_k \ne a,b$. Hence,
  $\ft_{a,b}$ is a function that is holomorphic and has no zeros in a neighbourhood $V_{a,b}$ of $[a,b]$. \abstand Therefore, locally
 $y_{a,b}(x)$ is an analytic continuation of $y(x)$ to some sheet of $\cu$ and can be used for numerical integration of differential forms. \abstand
  From our affine equation $\caff$ \eqref{eq:aff_model} it follows that other analytic branches above $V_{a,b}$ are given by
  \begin{align}
      \yab \mapsto \zeta^l \yab, \quad l = 0,\dots,m-1\,.
  \end{align}
  In our situation, we can assume that moving between sheets corresponds to multiplication of $\yab$ by $\zeta^{-1}$\,. 


  
  
  
  \newpage
  
  \subsection{Differential forms}
    The computation of the period matrix and the Abel-Jacobi map requires a basis of $\hd$ as $\C$-vector space. In this section we provide a basis that only 
   depends on $m$ and $d$ and
   is suitable for double-exponential integration.
%(see e.g. \cite{MM2009})
 One can show that the classes of the differentials in
	$$\WM = \left\{ \, \omega_{i,j}  \, \right\}_{\substack{1 \le i \le d-1, \\ 1 \le j \le m-1}} \quad \text{with} \quad \omega_{i,j} = \frac{ \dx^i}{i y^j}\,,$$
   form a $\C$-basis of $\coho$.
     
     \begin{prop}\label{prop:SEDFF}
	The differentials in
	\begin{align*}
	  \W =  \Big\{ \, \omega_{i,j} \in \WM \, \mid \, -Ni + jd - \gcd(m,d) \ge 0 \, \Big\}\,.
	\end{align*}
	form  a $\C$-basis of $\hd$.
     \end{prop}
     \begin{proof}
      First we show that the differentials in $\W$ are holomorphic.
      Let $\w_{i,j} = x^{i-1}y^{-j} \dx \in \WM$ and $\delta = \gcd(m,d)$. Then there are $\delta$ points  $P^{\infty}_1,\dots,P^{\infty}_{\delta} \in \cu$ above infinity. We write down the relevant divisors
      \begin{align*}
       \div(x) & = \sum_{k=1}^m \left(0,\zeta^k{f(0)}^{1/m}\right) - \frac{m}{\delta} \cdot \sum_{l = 1}^{\delta} P^{\infty}_l\,, \\
       \div(y) & = \sum_{k = 1}^d (x_k,0) - \frac{d}{\delta} \cdot \sum_{l = 1}^{\delta}  P^{\infty}_l\,, \\
       \div(\dx) & = (m-1)\sum_{k = 1}^d (x_k,0) - \left(\frac{m}{\delta} + 1\right)\cdot \sum_{l = 1}^{\delta}  P^{\infty}_l\,.
      \end{align*}
     Putting together the information, for $P \in \cu$ lying over $x_0 \in \P^1_{\C}$, we obtain
     \begin{align}\label{eq:diff_cases}
      v_P(\omega_{i,j}) & = (i-1) v_P(x) + v_P(\dx)  - j v_P(y) = 
	\begin{cases}
	 \ge 0\,, \hfill \text{if} \; x_0 \ne x_k,\infty \\
	 \ge m-1-j \ge 0 \,, \hfill \text{if} \; x_0 = x_k \\
	 = \frac{1}{\delta}(-mi-\delta+jd)\,,\hfill \text{if} \; x_0 = \infty
	\end{cases}\,.
     \end{align}
     We conclude: $\omega_{i,j} \in \WM$ is holomorphic if and only if $\omega_{i,j} \in \W$. \abstand
     Since the differentials in $\W$ are clearly $\C$-linearly independent, it remains to show that
     there are enough of them, i.e. $\#\W = g$. \abstand
     Counting the elements in $\W$ corresponds to counting lattice points $(i,j) \in \Z^2$ in the trapezoid given by the faces
     \begin{align*}
	1 \le i \le d-1\,,\\
	1 \le j \le m-1\,, \\
	i \le \frac{d}{m}j - \frac{\delta}{m}\,.
     \end{align*}
%      \subfile{images/ikz_pic_7}
      \begin{figure}[H]
      \begin{center}
	  % Tikz File 'tikz_pic_7.tex'
%\documentclass{standalone}
% \usepackage{tikz}
\usetikzlibrary{arrows}
\usetikzlibrary{shapes.misc}
\usetikzlibrary{decorations.markings}
\tikzset{cross/.style={cross out, draw=black, minimum size=2*(#1-\pgflinewidth), inner sep=0pt, outer sep=0pt},cross/.default={1pt}}
\tikzset{
    halfarrow1/.style={postaction={decorate},
        decoration={markings,mark=at position .5 with
        {\arrow[line width=0.4mm]{>}}}} }
\tikzset{
    halfarrow2/.style={postaction={decorate},
        decoration={markings,mark=at position .5 with
        {\arrow[line width=0.4mm]{<}}}} }
%\begin{document}
\begin{tikzpicture}
%      \draw (-1.8,2) node {$a$};  \draw (1.8,2) node {$b$};
%      \draw (-2.5,3) -- (-1.8,3);  \draw (1.8,3) -- (2.5,3); \filldraw [gray] (-1.8,3) circle (2pt); \filldraw [gray] (1.8,3) circle (2pt);
% \draw (-2.3,3) .. controls (-0.9,2.5) and (-0.9,2.5) .. (0,3) [halfarrow2];
% \draw (0,3) .. controls (0.9,3.5) and (0.9,3.5) .. (2.3,3) [halfarrow2];
% 
%      
%    \draw (-2.5,1) -- (-1.8,1);  \draw (1.8,1) -- (2.5,1); \filldraw [gray] (-1.8,1) circle (2pt); \filldraw [gray] (1.8,1) circle (2pt);
% \draw (-2.3,1) .. controls (-0.9,1.5) and (-0.9,1.5) .. (0,1) [halfarrow1];
% \draw (0,1) .. controls (0.9,0.5) and (0.9,0.5) .. (2.3,1) [halfarrow1];

\draw (-0.7,0) -- (8,0) [->]; \draw (8.5,0) node {$j$};
\draw (0,-0.7) -- (0,4) [->]; \draw (0,4.5) node {$i$};
\draw (1,-0.05) -- (1,0.05); \draw (1,-0.4) node {$1$};
\draw (2,-0.05) -- (2,0.05); \draw (2,-0.4) node {$2$};
\draw (3,-0.05) -- (3,0.05); \draw (3,-0.4) node {$3$};
\draw (4,-0.05) -- (4,0.05); \draw (4,-0.4) node {$4$};
\draw (5,-0.05) -- (5,0.05); \draw (5,-0.4) node {$5$};
\draw (6,-0.05) -- (6,0.05); \draw (6,-0.4) node {$6$};
\draw (7,-0.05) -- (7,0.05); \draw (7,-0.4) node {$7$};
\draw (-0.05,1) -- (0.05,1); \draw (-0.4,1) node {$1$};
\draw (-0.05,2) -- (0.05,2); \draw (-0.4,2) node {$2$};
\draw (-0.05,3) -- (0.05,3); \draw (-0.4,3) node {$3$};
\draw[gray] (1,1) -- (1,3);
\draw[gray] (1,1) -- (7,1);
\draw[gray] (7,1) -- (7,3);
\draw[gray] (1,3) -- (7,3);
\draw[gray] (0,-0.5) -- (8,3.5);
%\filldraw (1,1) circle (1pt); \draw [fill=white] (1,1) circle[radius= 0.9pt];
\filldraw [gray] (1,1) circle (1pt);
\filldraw [gray] (1,2) circle (1pt);
\filldraw [gray] (1,3) circle (1pt);
\filldraw [gray] (2,1) circle (1pt);
\filldraw [gray] (2,2) circle (1pt);
\filldraw [gray] (2,3) circle (1pt);
\filldraw (3,1) circle (1pt);
\filldraw [gray] (3,2) circle (1pt);
\filldraw [gray] (3,3) circle (1pt);
\filldraw (4,1) circle (1pt);
\filldraw [gray] (4,2) circle (1pt);
\filldraw [gray] (4,3) circle (1pt);
\filldraw (5,1) circle (1pt);
\filldraw (5,2) circle (1pt);
\filldraw [gray] (5,3) circle (1pt);
\filldraw (6,1) circle (1pt);
\filldraw (6,2) circle (1pt);
\filldraw [gray] (6,3) circle (1pt);
\filldraw (7,1) circle (1pt);
\filldraw (7,2) circle (1pt);
\filldraw (7,3) circle (1pt);
\end{tikzpicture}
%\end{document}

      \end{center}
    \caption{Counting points for $d =4$ and $m = 8$, we have $\delta = 4$ and therefore $g = 8$.} 
    \label{fig:holom_diff}
\end{figure}
     Carefully analyzing the situation at the vertices of the trapezoid, we find the following formula that counts the points.
     \begin{align}\label{eq:r_j}
	\#\W & = \sum_{j = 1}^{m-1} \floor*{\frac{d}{m}j - \frac{\delta}{m}} = \sum_{j = 1}^{m-1} \frac{dj-\delta-r_j}{m} =
	 \frac{d}{m} \sum_{j=1}^{m-1} j - \frac{m-1}{m}\delta - \frac{1}{m} \sum_{j=1}^{m-1} r_j\,,
     \end{align}
      where $r_j := dj - \delta \, \bmod m$. \abstand
      We claim that
      \begin{align}\label{eq:r_j2}
       \sum_{j=1}^{m-1} r_j = \frac{1}{2}(m^2 - (\delta+2)m + 2\delta)\,.
      \end{align}
      In order to show this, let $l := \frac{m}{\delta}$. First we note that $r_j = r_{j+l}$:
      $$r_{j+l} = d(j+l) - \delta \, \bmod m = dj + \frac{d}{\delta}m - \delta \, \bmod m =  dj - \delta \, \bmod m =  r_j$$
      and hence 
      \begin{align}\label{eq:r_j3}
       \sum_{j=1}^{m-1} r_j = \delta \cdot \sum_{j=1}^{l} r_j - r_m = \delta \cdot \sum_{j=1}^{l} r_j - (-\delta + m)\,.
      \end{align}
      Furthermore, $r_j$ can be written as multiple of $\delta$:
      $$r_j = \delta \left(\frac{d}{\delta}j - 1\right) \, \bmod m\,.$$
      From $\gcd(\frac{d}{\delta},l) = 1$ we conclude $\left\{ \, \frac{d}{\delta}j - 1 \, \bmod l  \mid \, 1 \le j \le l \, \right\} = \{ \, 0,\dots,l-1 \, \}$.
      Therefore,  
      \begin{align}\label{eq:r_j4}
       \sum_{j = 1}^l r_j = \sum_{j = 0}^{l-1} \delta j = \delta \cdot \frac{l(l-1)}{2}
      \end{align}
      and thus (\ref{eq:r_j3}) and (\ref{eq:r_j4}) imply
      $$\sum_{j=1}^{m-1} r_j = \delta \cdot \sum_{j=1}^{l} r_j + \delta - m = \delta^2 \cdot \frac{l(l-1)}{2} + \delta - m = \frac{1}{2}(m^2 - (\delta+2)m + 2\delta)\,.$$
      Plugging (\ref{eq:r_j2}) into (\ref{eq:r_j}) yields
      \begin{align*}
	\#\W & = \frac{d}{m}\frac{m(m-1)}{2} - \frac{m-1}{m}\delta - \frac{m^2 - (\delta+2)m + 2\delta}{2m} \\
	       & = \frac{d(m-1)}{2} - \delta  - \frac{m}{2} + \frac{\delta}{2} + 1 =  \frac{d(m-1) - (m-1) - \delta + 1}{2} \\
	       & = \frac{1}{2}((d-1)(m-1)-\delta+1) = g\,.
      \end{align*}
     \end{proof}

     \begin{rmk}
      Note that from \eqref{eq:diff_cases} it follows that the meromorphic differentials in $\WM$ are homolorphic at every finite place. 
     \end{rmk}
  
  \subsection{Cycles and homology}
    Content: - Definition of $\cyab$ and different representations, e.g. limit cycle
	     - Spanning tree
  
  \newpage
  
  \begin{thm}\label{thm:gen_set}
   The cycles $C = \left\{ \, \gamma_{e}^{(l)} \, \mid \, l \in \Z/m\Z, \, e \in E \, \right\}$ generate $\homo$.
  \end{thm}
  \begin{proof}
  Denote by $\alpha_x \in \pi(\P^1 \setminus X)$ a closed path that encircles the branch point $x \in X$ exactly once. Then,  due to the relation $1 = \prod_{x \in X} \alpha_x$,\,
  $\pi(\P^1 \setminus X)$ is freely generateted by $\{ \alpha_x \}_{x \in X'}$, where $X' = X \setminus \{ x_0 \}$, for any $x_0 \in X$. If $m \nmid d$, we choose to omit $x_0 = \infty$. \abstand
  Since our covering is cyclic, we have that $
  \pi_1(\cu \setminus \pr^{-1}(X)) = \ker(\pi(\P^1 \setminus X) \overset{\Phi}{\To} \Aut(\cu \setminus \pr^{-1}(X)))$. Moreover, $\Aut(\cu \setminus \pr^{-1}(X)) = C_m 
  \subset S_m$
  and $\Phi(\alpha_x) = (1 \dots m)$ for all $x \in X'$. Hence, for every word $\alpha = \alpha_1^{s_1}\dots \alpha_n^{s_n} \in \pi_1(\P^1 \setminus X)$ we have that
  $\alpha \in \ker(\Phi) \eq \sum_{i=1}^n s_i = 0 \bmod m$. \abstand
  Claim: $\pi_1(\cu \setminus \pr^{-1}(X)) = < \alpha_x^{-s} \alpha_y^{s}, \, \alpha_x^m  \, \mid \, l \in \Z, x,y \in X' >$\,. \\
  Proof: Induction on $n$\,. For $\alpha = \alpha_1^{s_1}$ $m$ divides $s_1$ and therefore $\alpha$ is generated by $\alpha_1^m$. For $n > 1$ we write
  $\alpha = \alpha_1^{s_1}\dots \alpha_n^{s_n} = (\alpha_1^{s_1} \dots \alpha_{n-1}^{s_{n-1}+s_n})(\alpha_{n-1}^{-s_n}\alpha_n^{s_n})$. \abstand
  We obtain the fundamental group of $\cu$ as
  $\pi(\cu) = \pi_1(\cu \setminus \pr^{-1}(X)) / < \alpha_x^{e_x} \, \mid \, x \in X >$, which is generated by 
  $< \alpha_x^{-s} \alpha_y^{s} \, \mid \, s \in \Z/m\Z, \, x,y \in X' >$. (In the case where $\delta \ne 1$, filling in the points above infinity will create additional relations among 
  these generators:
  $1 = (\prod_{x \in X'} \alpha_x)^{k \cdot e_{\infty}}$, $k = 1,\dots,\delta$\,.) \abstand
  All branch points $x,y \in X'$ are connected by a path $(x,e_1,\dots,e_t,y)$ in the spanning tree, so we can write $\alpha_x^{-s} \alpha_y^{s} = (\alpha_x^{-s}\alpha_{e_1}^{s})
  (\alpha_{e_1}^{-s}\alpha_{e_2}^{s})\dots(\alpha_{e_{t-1}}^{-s}\alpha_{e_t}^{s})(\alpha_{e_t}^{-s}\alpha_y^{s})$ and hence we have that 
  $\pi_1(\cu) = < \alpha_x^{-s} \alpha_y^{s} \, \mid \, s \in \Z/m\Z, \, (x,y) \in E >$\,. \abstand
    All that is left to see, is that for each $e = (x,y) \in E$ there exists a bijection $\varphi_e$ between
  $\{ \, \gamma_e^{(l)} \, \mid \, l \in \Z/m\Z \, \} \overset{\varphi_e}{\longleftrightarrow} \{ \, \alpha_x^{-s} \alpha_y^{s} \, \mid \, s \in \Z/m\Z \, \}$\,: \abstand
  If we choose basepoints $p_0 \in P^1 \setminus X$ for $\pi_1(P^1 \setminus X)$ and $P_0 \in \pr^{-1}(p_0)$ for $\pi_1(\cu \setminus \pr^{-1}(X))$ and $\pi_1(\cu)$ respectively.
  Depending on the choice of $P_0$, there exist $l_0, s_0 \in \Z/m\Z$ such that $\gamma_e^{(l_0)}$ is homotopic to $\alpha_x^{-s_0} \alpha_y^{s_0}$ in $\pi_1(\cu,P_0)$. This induces the bijection
  $\varphi_e(l_0+l) = s_0+s$\,. 
%   \todo Include $\infty$.\\
%    Each cycle $\alpha \in \homo$ can be encoded as $\alpha = \sum_{k = 1}^d n_kx_k$ with $n_1 + \dots n_d = 0 \bmod m$.
%    Here $n_k \in \Z/m\Z$ indicates that $\alpha$ encircles the branch point $x_k$ $n_k$-times in positive orientation 
%    on neighbouring sheets and the
%    condition on the $n_k$ ensures that $\alpha$ is a closed path.
%    For $e = (a,b)$ we can now write $\gamma_e^{(l)} = a - b$ (Note: $n \cdot \gamma_e^{(l)} = a-b$) , which implies $\sum_{l = 1}^n \gamma_e^{(l-1)} = n(a-b)$, 
%    for $n = 1,\dots,m-1$. \\
%    Since $E$ is a spanning tree, every cycle $x_i - x_j$ involving only two branch points can be written as sum of elements in the spanning tree,
%    i.e. $x_i - x_j = \sum_{e \in E} c_{i,j,e} (x_{e_1}-x_{e_2})$ with $c_{i,j,e} \in \{ \pm 1, 0 \}$. Writing
%    $n_d = -(n_1 + \dots n_{d-1}$) and
%    combining the information, we obtain
%    \begin{align*}
%     \alpha \; & = \; n_1x_1 + \dots n_dx_d = n_1(x_1 - x_d) + \dots + n_{d-1}(x_{d-1} - x_d) \\
%    &  = \;  n_1 \sum_{e \in E} c_{1,d,e} (x_{e_1}-x_{e_2}) + \dots + n_{d-1} \sum_{e \in E} c_{d-1,d,e} (x_{e_1}-x_{e_2}) \\
%    & = \; \sum_{e \in E} c_{1,d,e} n_1 (x_{e_1}-x_{e_2}) + \dots + c_{d-1,d,e} n_{d-1} (x_{e_1}-x_{e_2}) \\
%   &  = \; \sum_{e \in E}  \left( c_{1,d,e} \sum_{l = 1}^{n_1} \gamma_e^{(l)} + \dots +  c_{d-1,d,e} \sum_{l = 1}^{n_{d-1}} 
%   \gamma_e^{(l)} \right) \in \; < C >\,.
% %   & = \sum_{e \in E}  \left( c_{1,d,e} \sum_{l = 1}^{n_1} + \dots + c_{d-1,d,e} \sum_{l = 1}^{n_{d-1}} \right)
% %  \gamma_e^{(l)} 
%    \end{align*}
  \end{proof}


  
  
  
\end{document}