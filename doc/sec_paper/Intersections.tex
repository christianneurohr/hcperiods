\documentclass[main.tex]{subfiles}

\begin{document}

  \section{Intersections}\label{sec:intersections}

  Content: - Def. shifting number + Thm. shifting number
	   - 4 big Lemma's for intersections
	   - Building the intersection matrix from blocks + algorithm proves rank 2g
	   - Long proof of Thm. shifting number

	In this section we explain how to compute the intersection numbers between the cycles in $C$. 
	
	\begin{defn}\label{def:shift_numb}
	Let $\cyab, \cycd \in C$ be elementary cycles and let $V := V_{a,b} \cap V_{c,d}$. Then, $\yab$ and $\yab$ are analytic on $V$ and for all $x$
	$s \in \Z/m\Z$ such that
	\begin{align}
	 \frac{\yab(x)}{\ycd(x)} = \zeta^s \quad \text{for all} \; x \in V\,.
	\end{align}
	\end{defn}

	 
	    We can compute the shiftung number for all elementary cycles $\cyab, \cycd \in C$:
    
    \begin{thm}\label{thm:shift_numb}
    Assume $[a,b] \cap [c,d] = \{x\}$. With notations as above and $\tau = \arg \left( \frac{d-c}{b-a} \right)$ we have that
     \begin{align}\label{eq:shift_numb}
       s(\alpha,\beta) = \frac{1}{2\pi} \left( \theta +  m \cdot \arg \left( \frac{(b-a)^{\frac{d}{m}} \ft_{a,b}(x)}{(d-c)^{\frac{d}{m}} \ft_{c,d}(x)} \right) \right) \in \Z/m\Z\,,
     \end{align}
     where
      \begin{align*}\theta = \begin{cases}
                              \, -\pi+\tau, \quad \text{if} \quad \tau < 0 \quad \text{and} \quad (\,x = b = c \quad \text{or} \quad x = a = d\,)\,,\\
                              \quad\! \pi-\tau, \quad \text{if} \quad \tau \ge 0 \quad \text{and} \quad (\,x = b = c \quad \text{or} \quad x = a = d\,)\,,\\
                              \hspace{0.7cm} -\tau, \quad \text{if} \quad x = a = c \quad \text{or} \quad x = b = d\,,\\
                              \hspace{1cm} 0, \quad \text{if} \quad x  \in \; ]a,b[ \,\cap\, ]c,d[\,.
                             \end{cases}
     \end{align*}
    \end{thm}
    \begin{proof}
     See \S \todo ref.
    \end{proof}

     Using this information, we obtain the intersection numbers graphically by using visualizations of $\alpha$ on $\beta$ running on different sheets.
     However, if we find homologous cycles, whose projections do not intersect, their lifts to the surface cannot intersect.
    
%     At a common point $p$, that is no branch points, the notion of sheets makes sense and there exists an $s \in \Z/N\Z$ such that\,. This means that above the point $p$
%      $\beta_0$ runs $s$ sheets above $\alpha_0$. We will call $s = s(\alpha,\beta)$ the \textit{shifting number} of $\alpha$ and $\beta$ at $p$.
%     If $p$ is a branch point, we choose homologous paths whose $x$-projections intersect at a non-branch point and take a limit.
    
    \newpage
    \begin{thm}\label{thm:intsec_numb}
      Let $\cyab$, $\cycd$ be elementary cycles, $\tau = \arg\left(\frac{d-c}{b-a}\right)$ and let $s=s(\cyab,\cycd,x)$ be the
     shifting number at a point $x$. Then the intersection numbers are given by
      \begin{align}
       \begin{split}
        (\,\cyab^{(l)} \circ \cycd^{(k)}\,) = 
        \begin{cases}
            \hspace*{0.95cm} -1, \quad & \text{if} \quad s=0 \quad \text{and} \quad (\,x = b = c \quad \text{or} \quad x = a = d\,)\,,\\
	     \hspace*{0.95cm} +1, \quad &\text{if} \quad s=-1 \quad \text{and} \quad (\,x = b = c \quad \text{or} \quad x = a = d\,)\,,\\
          \quad\! \sgn(\tau), \quad & \text{if} \quad s=0 \quad \text{and} \quad (\,x = a = c \quad \text{or} \quad x = b = d\,)\,,\\
            -\sgn(\tau), \quad & \text{if} \quad s=-\sgn(\tau) \quad \text{and} \quad (\,x = a = c \quad \text{or} \quad x = b = d\,)\,,\\
          \hspace*{0.95cm} +1, \quad &\text{if} \quad s=0 \quad \text{and} \quad [a,b] = [c,d]\,,\\
          \hspace*{0.95cm} -1, \quad &\text{if} \quad s=-1 \quad \text{and} \quad [a,b] = [c,d]\,,\\
         \hspace*{0.95cm}  \quad\! 0, \quad &\text{otherwise}\,.
        \end{cases}
       \end{split}
      \end{align}
    \end{thm}
	   
\end{document}