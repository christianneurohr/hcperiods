\documentclass[main.tex]{subfiles}

\begin{document}

  \newpage

  \section{Outlook}\label{sec:outlook}

  In this paper we presented an approach based on numerical integration for
  multiprecision computation of period matrices and the Abel-Jacobi map of
  superelliptic curves given by $m > 1$ and squarefree $f \in \C[x]$.
 
  Integration along a spanning tree and the special geometry of such curves
  make it possible to compute these objects too high precision performing only
  a few numerical integrations. The resulting algorithm has an excellent
  scaling with the genus and works for several thousand digits of precision.

  \subsection{Reduced small period matrix}

   For a given curve our algorithm computes a small period matrix
   $\tau$ in the Siegel upper half-space $\mathcal{H}_g$ which is arbitrary
   in the sense that it depends on the choice of a symplectic basis made
   during the algorithm.
   
   For applications like the computation of theta functions it is useful to
   have a small period matrix in the Siegel fundamental domain $\mathcal{F}_g \subset
   \mathcal{H}_g$ (see \cite[\S 1.3]{PlaneQuarticsCM}).
  
   We did not implement any such reduction.
   The authors of \cite{PlaneQuarticsCM} give a theoretical sketch of
   an algorithm (Algorithm 1.9) that achieves this reduction step, as well as
   two practical versions (Algorithms 1.12 and 1.14) which work in any genus and have been implemented for $g
   \le 3$. It would be interesting to combine this with our implementation.
  
  \subsection{Generalizations}
  
  We remark that there is no theoretical obstruction to generalizing our
  approach to more general curves. In a first step the algorithm could be
  extended to all complex superelliptic curves given by $m > 1$ and $f \in
  \C[x]$, where $f$ can have multiple roots of order at most $m-1$.
  Although several adjustments would have to be made (e.g.\ differentials,
  homology, integration), staying within the superelliptic setting promises
  a fast and rigorous extension of our algorithm. 
  
  \medskip
  
  We also believe that the strategy employed here (numerical integration between
  branch points combined with information about local intersections) could
  be adapted to completely general algebraic curves given by $F \in \C[x,y]$.
  However, serious issues have to be overcome:
  \begin{itemize}
      \item On the numerical side we no longer have a nice $m$-th root function, it may be replaced by
          Newton's method between branch points (analytic continuation has to be performed on all sheets) and
          Puiseux series expansion around them.
      \item On the geometric side we cannot easily define loops, so that given a set
          of ``half'' integrals each connecting two branch points, we need to combine them in order
          to obtain all at once true loops and a symplectic basis.
          An appropriate notion of shifting number and local intersection is needed here,
          as well as a combination technique.
  \end{itemize}
  We did not investigate further: at this point the advantages
  of superelliptic curves which are utilized by our approach are already lost
 (simple geometry of branch points and $m-1$ integrals at the cost of one), 
  so it is
  not clear whether this approach might be more efficient than other methods.

  \biblio
  \end{document}
