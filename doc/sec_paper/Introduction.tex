\documentclass[main.tex]{subfiles}

\begin{document}

  Content: Problem formulation, Applications, Existing work, Thm: Complexity Abel-Jacobi

  \section{Introduction}

  The Abel-Jacobi map links a curve to a complex torus.
  In particular the matrix of periods makes it possible to define the Riemann
  theta function of the curve, which is an object of central interest in
  mathematics: let us
  mention the theory of abelian function or integration of partial differential
  equations.

  In the context of cryptography and number theory, periods also appear
  in the BSD conjecture or as a tool to identify isogenies or to find
  curves having prescribed complex multiplication \cite{vanWamelen06}.
  For such diophantine applications, it is necessary to compute
  integrals to large precision (say thousand digits), and to have
  rigorous results.

  \subsection{Existing algorithms}

  For genus 1 and 2, methods based on isogenies (AGM \cite{CremonaAGM13},
  Richelot \cite{BostMestre88}, Borchards means \cite{Labrande16})
  make it possible to compute periods to arbitrary precision in almost
  linear time. However these techniques scale very badly when the genus grows.

  For modular curves, the modular symbols machinery and termwise integration of
  modular forms expansions also give excellent algorithms
  \cite[sec 3.2]{Mascot13}.

  For general hyperelliptic curve of arbitrary genus, the Magma implementation
  due to van Wamelen \cite{vanWamelen06} computes this map.
  However it is limited in terms of precision (less
  than 2000 digits) and some bugs are experienced on
  certain configurations of branch points. This implementation motivated our
  work. Using a different strategy
  (integration along a tree instead of around Voronoi cells)
  we obtain a much better algorithm and rigorous results.

  For completely general algebraic curves, there is an implementation in Maple
  due to van Hoeij and Deconinck \cite{DeconinckvanHoeij01}.
  We found that this package is not suitable for high precision purposes.

  \subsection{Main result}

  This paper adresses the problem of computing the period matrix and the
  Abel-Jacobi on curves of the form $y^m = f(x)$, which generalize immediately
  hyperelliptic curves and are usually called superelliptic.

  We take advantage of their specific geometry to obtain the following
  (see Theorem \ref{thm:complexity_integrals})
  \begin{thm}
      Let $y^m=f(x)$ be a superelliptic curve of genus $g$,
      where $f$ has degree $d$.
      We can compute a basis of the period lattice to precision
      $D$ digits using $O(n(g+\log D)D^2\log^{2+\varepsilon} D)$
      binary operations, where $\epsilon>0$ is chosen so that
      the multiplication of precision $D$ numbers has complexity
      $O(D\log^{1+\epsilon}D)$.
  \end{thm}

  This part only involves numerical integration of differentials. We remark
  that this complexity is very favorable in terms of the genus.

  In order
  to obtain standard periods we rely on some linear algebra to obtain a
  symplectic basis of homology (see Section \ref{sec:symp_basis}).
  In practice this step has negligible complexity
  and produces a base change with tiny coefficients. However we have no proof
  of this fact which has to do with the very specific form of our input,
  so we leave it as a supplementary hypothesis to claim the
  following
  \begin{thm}
      Let $y^m=f(x)$ be a superelliptic curve of genus $g$ whose periods
      have been computed.
      Provided the symplectic reduction
      step runs in $O(g^2)$ operations and produces $O(1)$ coefficients,
      we compute the reduced period matrix $τ$ to precision $D$
      digits using an extra $O(g^{2.8}D\log^{1+\varepsilon}D)$ operations.
  \end{thm}

  \subsection{Rigorous implementation}

  The algorithm has been implemented in C using the Arb library \cite{Johansson2013arb}.
  This system represents a complex numbers as a floating point approximation
  plus an error bound, and automatically
  takes into account all precision loss occurring through the
  execution of the program. With this model we can certify
  the accuracy of the numerical results of our algorithm (up to human or even
  compiler errors, as usual).

  Another implementation has been done in Magma. Both are publicly available
  on github \url{https://github.com/pascalmolin/hcperiods}.

  \subsection{Acknowledgements}

  The first author wants to thank the crypto team at Inria Nancy, where
  a first version of this work was carried out in the case of hyperelliptic
  curves.

\end{document}
