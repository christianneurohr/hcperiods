\documentclass[main.tex]{subfiles}

\begin{document}

  Content: Problem formulation, Applications, Existing work, Thm: Complexity Abel-Jacobi

  \section{Introduction}

  The numerical evaluation of integrals defined on algebraic curves
  is a problem of long interest. Being able to compute such integrals
  rigorously and to arbitrary precision is of crucial interest in
  number theory.

  \subsection{Existing algorithms}

  For genus 1 and 2, methods based on isogenies (AGM, Richelot \cite{BostMestre88}, Borchards means \cite{Labrande16})
  make it possible to compute periods to arbitrary precision in almost linear time.

  For modular curves, the modular symbols machinery and termwise integration of
  modular forms expansions also give excellent algorithms
  \cite[sec 3.2]{Mascot13}.

  For general hyperelliptic curve of arbitrary genus, the Magma implantation
  due to P. Van Wamelen \cite{vanWamelen06} computes this map.
  However it is limited in terms of precision (less
  than 2000 digits) and some bugs are experienced on
  certain configurations of branch points. This implementation motivated our
  work. Using a different strategy
  (integration along a tree instead of around Voronoi cells)
  we obtain a much better algorithm and rigorous results.

  For completely general algebraic curves, there is an implementation in Maple
  due to van Hoeij and Dekoninck \cite{DeconinckvanHoeij01}.
  We found that this package is not suitable for high precision purposes.

\end{document}
