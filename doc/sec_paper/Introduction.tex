\documentclass[main.tex]{subfiles}

\begin{document}

  \section{Introduction}

  The Abel-Jacobi map links a curve to a complex torus.
  In particular the matrix of periods makes it possible to define the Riemann
  theta function of the curve, which is an object of central interest in
  mathematics and physics: let us
  mention the theory of abelian functions or integration of partial differential
  equations.

  In the context of cryptography and number theory, periods also appear
  in the BSD conjecture or as a tool to identify isogenies or to find
  curves having prescribed complex multiplication \cite{vanWamelen06}.
  For such diophantine applications, it is necessary to compute
  integrals to large precision (say thousand digits) and to have
  rigorous results.

  \subsection{Existing algorithms}

  For genus 1 and 2, methods based on isogenies (AGM \cite{CremonaAGM13},
  Richelot \cite{BostMestre88}, Borchardt mean \cite{Labrande16})
  make it possible to compute periods to arbitrary precision in almost
  linear time. However, these techniques scale very badly when the genus grows.

  For modular curves, the modular symbols machinery and termwise integration of
  expansions of modular forms give excellent algorithms
  \cite[\S 3.2]{Mascot13}.

  For hyperelliptic curves of arbitrary genus, the Magma implementation
  due to van Wamelen \cite{vanWamelen06} computes this map.
  However, it is limited in terms of precision (less
  than 2000 digits) and some bugs are experienced on
  certain configurations of branch points. The shortcomings of this implementation motivated our
  work. Using a different strategy
  (integration along a tree instead of around Voronoi cells)
  we obtain a much faster, more reliable algorithm and rigorous results.

  For general algebraic curves, there is an implementation in Maple
  due to Deconinck and van Hoeij \cite{DeconinckvanHoeij01}.
  We found that this package is not suitable for high precision purposes.

  We also mention the Matlab implementations due to Frauendiener and Klein for hyperelliptic curves \cite{FrauendienerKlein2015}
  and for general algebraic curves \cite{FrauendienerKlein2011}.

  \subsection{Main results}

  This paper adresses the problem of computing period matrices and the
  Abel-Jacobi map of algebraic curves given by an affine equation of the form  (see Definition \ref{m-def:se_curve})
  \begin{equation*}
  y^m = f(x), \quad m > 1, f \in \C[x] \text{ separable of degree} \deg(f) = n \ge 3.
  \end{equation*}
  They generalize
  hyperelliptic curves and are usually called \textit{superelliptic curves}.

  We take advantage of their specific geometry to obtain the following
  (see Theorem \ref{m-thm:complexity_integrals})
  \begin{thm}
      Let $\cu$ be a superelliptic curve of genus $g$ defined by the equation an $y^m=f(x)$
      where $f$ has degree $n$.
      We can compute a basis of the period lattice to precision
      $D$ digits using $$O(n(g+\log D)D^2\log^{2+\varepsilon} D) \text{ binary operations,}$$
     where $\epsilon>0$ is chosen so that
      the multiplication of precision $D$ numbers has complexity
      $O(D\log^{1+\epsilon}D)$.
  \end{thm}

  This part only involves numerical integration of differentials. We remark
  that this complexity is very favorable in terms of the genus.

  In order
  to obtain standard periods we rely on some linear algebra to obtain a
  symplectic basis of the homology group (see Section \ref{m-subsec:symp_basis}).
  In practice this step takes negligible time
  and produces a base change with tiny coefficients. However, we have no proof
  of this fact which has to do with the very specific form of our input,
  so we leave it as a supplementary hypothesis to claim the
  following
  \begin{thm}
     Let $\cu$ be a superelliptic curve of genus $g$ defined by the equation an $y^m=f(x)$ whose periods
      have been computed.
      Provided the symplectic reduction
      step runs in $O(g^2)$ operations and produces $O(1)$ coefficients,
      we compute the small period matrix $τ$ to precision $D$
      digits using an extra $O(g^{2.8}D\log^{1+\varepsilon}D)$ operations.
  \end{thm}

  \todo Theorem on Abel-Jacobi map?


  \subsection{Rigorous implementation}
  \label{subsec:arb}

  The algorithm has been implemented in C using the Arb library \cite{Johansson2013arb}.
  This system represents a complex numbers as a floating point approximation
  plus an error bound, and automatically
  takes into account all precision loss occurring through the
  execution of the program. With this model we can certify
  the accuracy of the numerical results of our algorithm (up to human or even
  compiler errors, as usual).

  Another implementation has been done in Magma \cite{Magma}. Both are publicly available
  on github \url{https://github.com/pascalmolin/hcperiods}.

  \subsection{Interface with the LMFDB}

  Having rigorous period matrices is a valuable input for the methods developped by
  Sijsling et al. \cite{CMSVEndos} to compute endormorphism rings of Jacobians of hyperelliptic
  curves.
  During a meeting aimed at expanding the `L-functions and modular forms database' \cite[LMFDB]{lmfdb}
  to genus $3$ curves, the Magma implementation of our algorithm was incorporated in their framework
  to successfully compute the endomorphism rings of Jacobians of $67,879$ hyperelliptic
  curves of genus $3$, and confirm those of the $66,158$ genus
  2 curves that are currently in the database.

  For these applications big period matrices were computed to $300$ digits precision.

  \subsection{Structure of the paper}

  In Section \ref{m-sec:ajm} we briefly review the objects we are interested
  in, namely period matrices and the Abel-Jacobi map of nice algebraic curves.

  The ingredients to obtain these objects, a basis of holomorphic differentials
  and a homology basis, are made explicit in the case of superelliptic curves
  in Section \ref{m-sec:se_curves}.

  We give formulas for the computation of periods in Section
  \ref{m-sec:strat_pm} and explain how to obtain from them the standard period
  matrices using symplectic reduction.

  In Section \ref{sec:intersections} we give explicit formulas for the
  intersection numbers of our homology basis.

  For numerical integration we employ two different integration schemes that
  are explained in Section \ref{m-sec:numerical_integration}: the
  double-exponential integration and
  (in the case of hyperelliptic curves) Gauss-Chebychev integration.

  The actual computation of the Abel-Jacobi map is explained in detail in
  Section \ref{m-sec:comp_ajm}.

  In Section \ref{m-sec:comp_asp} we analyze the complexity of our algorithm
  and share some insights on the implementation.

  Finally, Section \ref{m-sec:examples_timings} contains some tables with
  running times to demonstrate the performance of the code.

  \subsection{Acknowledgements}

  The first author wants to thank the crypto team at Inria Nancy, where
  a first version of this work was carried out in the case of hyperelliptic
  curves.

  The second author wants to thank Steffen Müller and Florian Hess for helpful discussions.
  Moreover, he acknowledges the support from the DAAD under grant $57212102$.

\end{document}
